%\documentclass{article}
\documentclass[12pt]{article}
\usepackage{latexsym}
\usepackage{amsmath}
\usepackage{amssymb}
\usepackage{relsize}
\usepackage{geometry}
\geometry{letterpaper}

%\usepackage{showlabels}

\textwidth = 6.0 in
\textheight = 8.5 in
\oddsidemargin = 0.0 in
\evensidemargin = 0.0 in
\topmargin = 0.2 in
\headheight = 0.0 in
\headsep = 0.0 in
%\parskip = 0.05in
\parindent = 0.35in


%% common definitions
\def\stackunder#1#2{\mathrel{\mathop{#2}\limits_{#1}}}
\def\beqn{\begin{eqnarray}}
\def\eeqn{\end{eqnarray}}
\def\nn{\nonumber}
\def\baselinestretch{1.1}
\def\beq{\begin{equation}}
\def\eeq{\end{equation}}
\def\ba{\beq\new\begin{array}{c}}
\def\ea{\end{array}\eeq}
\def\be{\ba}
\def\ee{\ea}
\def\stackreb#1#2{\mathrel{\mathop{#2}\limits_{#1}}}
\def\Tr{{\rm Tr}}
\newcommand{\gsim}{\lower.7ex\hbox{$
\;\stackrel{\textstyle>}{\sim}\;$}}
\newcommand{\lsim}{\lower.7ex\hbox{$
\;\stackrel{\textstyle<}{\sim}\;$}}
%%%%%%%%%%
\newcommand{\nfour}{${\mathcal N}=4$ }
\newcommand{\ntwo}{${\mathcal N}=2$ }
\newcommand{\ntwon}{${\mathcal N}=2$}
\newcommand{\ntwot}{${\mathcal N}= \left(2,2\right) $ }
\newcommand{\ntwoo}{${\mathcal N}= \left(0,2\right) $ }
\newcommand{\ntwoon}{${\mathcal N}= \left(0,2\right) $}
\newcommand{\none}{${\mathcal N}=1$ }
%%%%%%%%%%%
\newcommand{\nonen}{${\mathcal N}=1$}
\newcommand{\vp}{\varphi}
\newcommand{\pt}{\partial}
\newcommand{\ve}{\varepsilon}
\newcommand{\gs}{g^{2}}
\newcommand{\qt}{\tilde q}
%\renewcommand{\theequation}{\thesection.\arabic{equation}}

%%
\newcommand{\p}{\partial}
\newcommand{\wt}{\widetilde}
\newcommand{\ov}{\overline}
\newcommand{\mc}[1]{\mathcal{#1}}
\newcommand{\md}{\mathcal{D}}

\newcommand{\GeV}{{\rm GeV}}
\newcommand{\eV}{{\rm eV}}
\newcommand{\Heff}{{\mathcal{H}_{\rm eff}}}
\newcommand{\Leff}{{\mathcal{L}_{\rm eff}}}
\newcommand{\el}{{\rm EM}}
\newcommand{\uflavor}{\mathbf{1}_{\rm flavor}}
\newcommand{\lgr}{\left\lgroup}
\newcommand{\rgr}{\right\rgroup}

\newcommand{\Mpl}{M_{\rm Pl}}
\newcommand{\suc}{{{\rm SU}_{\rm C}(3)}}
\newcommand{\sul}{{{\rm SU}_{\rm L}(2)}}
\newcommand{\sutw}{{\rm SU}(2)}
\newcommand{\suth}{{\rm SU}(3)}
\newcommand{\ue}{{\rm U}(1)}

%%%%%%%%%%%%%%%%%%%%%%%%%%%%%%%%%%%%%%%
%  Slash character...
\def\slashed#1{\setbox0=\hbox{$#1$}             % set a box for #1
   \dimen0=\wd0                                 % and get its size
   \setbox1=\hbox{/} \dimen1=\wd1               % get size of /
   \ifdim\dimen0>\dimen1                        % #1 is bigger
      \rlap{\hbox to \dimen0{\hfil/\hfil}}      % so center / in box
      #1                                        % and print #1
   \else                                        % / is bigger
      \rlap{\hbox to \dimen1{\hfil$#1$\hfil}}   % so center #1
      /                                         % and print /
   \fi}                                         %
%%EXAMPLE:  $\slashed{E}$ or $\slashed{E}_{t}$

%%

\newcommand{\LN}{\Lambda_\text{SU($N$)}}
\newcommand{\sunu}{{\rm SU($N$) $\times$ U(1) }}
\newcommand{\sunun}{{\rm SU($N$) $\times$ U(1)}}
\def\cfl {$\text{SU($N$)}_{\rm C+F}$ }
\def\cfln {$\text{SU($N$)}_{\rm C+F}$}
\newcommand{\mUp}{m_{\rm U(1)}^{+}}
\newcommand{\mUm}{m_{\rm U(1)}^{-}}
\newcommand{\mNp}{m_\text{SU($N$)}^{+}}
\newcommand{\mNm}{m_\text{SU($N$)}^{-}}
\newcommand{\AU}{\mc{A}^{\rm U(1)}}
\newcommand{\AN}{\mc{A}^\text{SU($N$)}}
\newcommand{\aU}{a^{\rm U(1)}}
\newcommand{\aN}{a^\text{SU($N$)}}
\newcommand{\baU}{\ov{a}{}^{\rm U(1)}}
\newcommand{\baN}{\ov{a}{}^\text{SU($N$)}}
\newcommand{\lU}{\lambda^{\rm U(1)}}
\newcommand{\lN}{\lambda^\text{SU($N$)}}
%\newcommand{\Tr}{{\rm Tr\,}}
\newcommand{\bxir}{\ov{\xi}{}_R}
\newcommand{\bxil}{\ov{\xi}{}_L}
\newcommand{\xir}{\xi_R}
\newcommand{\xil}{\xi_L}
\newcommand{\bzl}{\ov{\zeta}{}_L}
\newcommand{\bzr}{\ov{\zeta}{}_R}
\newcommand{\zr}{\zeta_R}
\newcommand{\zl}{\zeta_L}
\newcommand{\nbar}{\ov{n}}

\newcommand{\ssm}{{\scriptscriptstyle(M)}}
\newcommand{\sse}{{\scriptscriptstyle(E)}}
\newcommand{\cell}{{\mathcal L}}
\newcommand{\CPC}{CP($N-1$)$\times$C }
\newcommand{\CPCn}{CP($N-1$)$\times$C}
\newcommand{\cpn}{CP$(N-1)\,$}

\newcommand{\lar}{\lambda_R}
\newcommand{\lal}{\lambda_L}
\newcommand{\larl}{\lambda_{R,L}}
\newcommand{\lalr}{\lambda_{L,R}}
\newcommand{\blar}{\ov{\lambda}{}_R}
\newcommand{\blal}{\ov{\lambda}{}_L}
\newcommand{\blarl}{\ov{\lambda}{}_{R,L}}
\newcommand{\blalr}{\ov{\lambda}{}_{L,R}}

\newcommand{\tgamma}{\wt{\gamma}}
\newcommand{\btgamma}{\ov{\tgamma}}
\newcommand{\bpsi}{\ov{\psi}{}}
\newcommand{\bphi}{\ov{\phi}{}}
\newcommand{\bxi}{\ov{\xi}{}}

\newcommand{\ff}{\mc{F}}
\newcommand{\bff}{\ov{\mc{F}}}

\newcommand{\eer}{\epsilon_R}
\newcommand{\eel}{\epsilon_L}
\newcommand{\eerl}{\epsilon_{R,L}}
\newcommand{\eelr}{\epsilon_{L,R}}
\newcommand{\beer}{\ov{\epsilon}{}_R}
\newcommand{\beel}{\ov{\epsilon}{}_L}
\newcommand{\beerl}{\ov{\epsilon}{}_{R,L}}
\newcommand{\beelr}{\ov{\epsilon}{}_{L,R}}

\newcommand{\bi}{{\bar \imath}}
\newcommand{\bj}{{\bar \jmath}}
\newcommand{\bk}{{\bar k}}
\newcommand{\bl}{{\bar l}}
\newcommand{\bm}{{\bar m}}

\newcommand{\hsigma}{{\hat{\sigma}}}
\newcommand{\ww}{\tilde{\mc{W}}{}_\text{eff}}

\begin{document}

{\bf\centering
\large\underline{Difference of Superpotential in CP(2) at Large Real Masses}}
\vspace{1.0cm}

	The masses so far were taken ``real'', that is, one of the mass (call it $m_0$) sits
	on the real negative axis. 
	The original motivation for this choice is the AD point, for which this is the case.
	However, now $ |m| $ is taken large, $ | m | \,\gg\, \Lambda $.

	Again, denote
\beq
	\ww (\hsigma) ~~=~~ 4\pi \cdot \mc{W}_\text{eff}(\hsigma)\,,
\eeq
	such that
\beq
	\ww (\sigma_p) ~~=~~ -\, N\, \sigma_p ~-~ \sum\, m_j\, \ln (\sigma_p \,-\, m_j)\,.
\eeq

	I performed the expansion of the vacuum values $ \ww(\sigma_p) $ in $ \Lambda / m $.
	The vacua $ \sigma_p $ are chosen as
\begin{align}
%
\notag
	\sigma_0 & ~~=~~ \sqrt[3]{\Lambda^3 \,+\, m_0^3~} \,,  \\
%
	\sigma_1 & ~~=~~ \sqrt[3]{\Lambda^3 \,+\, m_0^3~} \, e^{2\pi i /3}\,, \\
%
	\sigma_2 & ~~=~~ \sqrt[3]{\Lambda^3 \,+\, m_0^3~} \, e^{- 2\pi i /3}\,.
\notag
\end{align}
	Note that $ m_0 $ is negative, therefore the cubic root (and hence $\sigma_0$) is a negative
	real number.
	The answer for the vacuum values is as follows (here $ m \,\equiv\, |m_0|$),
\begin{align}
%
\label{vval}
	\ww(\sigma_0) & ~~=~~ m_0 
	\lgr   3\,\ln \Bigl(\frac{m}{\Lambda}\Bigl) \,+\,
\Bigl(\frac{3}{2} \ln 3 \,+\, \frac{5}{6} \pi \sqrt{3} \,-\, 3 \Bigr)
\,+\, O \bigl( \Lambda/m_0 \bigr)^3 \rgr ,
	\\[2mm]
%
\notag
	\ww(\sigma_1) & ~~=~~ m_1 
	\lgr   3\,\ln \Bigl(\frac{m}{\Lambda}\Bigl) \,+\,
\Bigl(\frac{3}{2} \ln 3 \,-\, \frac{1}{6} \pi \sqrt{3} \,-\, 3 \,-\, i\pi \Bigr)
\,+\, O \bigl( \Lambda/m_0 \bigr)^3 \rgr ,
	\\[2mm]
%
\notag
	\ww(\sigma_2) & ~~=~~ m_2
	\lgr   3\,\ln \Bigl(\frac{m}{\Lambda}\Bigl) \,+\,
\Bigl(\frac{3}{2} \ln 3 \,-\, \frac{1}{6} \pi \sqrt{3} \,-\, 3 \,+\, i\pi \Bigr)
\,+\, O \bigl( \Lambda/m_0 \bigr)^3 \rgr .
\end{align}
	The structure of the answer is as follows. 
	The first term is the large logarithmic term, which Misha mentioned.
	This term is the same for all vacua.
	The second round bracket is a finite number. 
	It is {\it real} for the vacuum $ \sigma_0 $, and mutually conjugate for the other two vacua.
	Finally, $ O \bigl( \Lambda/m_0 \bigr)^3 $ represents corrections which go in powers
	of $ (\Lambda/m)^3 $.
	These corrections are in general complex.

	Now, at large $m$, the logarithmic term in each of values of \eqref{vval} dominates, and since
	it is the same for all vacua, then indeed,
\beq
	\ww(\sigma_k) ~~\propto~~ m_k\,,  \qquad\qquad \lim~m~\gg~\Lambda\,,
\eeq
	with a {\it real} coefficient of proportionality. 
	But that amounts to ignoring all finite terms in Eq.~\eqref{vval}.

	As for the kink masses, or the difference of the superpotential values, they come out to be
\beq
	\ww(\sigma_{k+1}) ~-~ \ww(\sigma_k)  ~~=~~ 
	(m_{k+1} \,-\, m_k)
	\lgr R ~+~ 3\, \ln \Bigl(\frac{m}{\Lambda}\Bigl) ~+~ O \bigl( \Lambda/m_0 \bigr)^3 \rgr,
\eeq
	where $ R $ is a finite contribution
\beq
	R ~~=~~ \frac{\pi}{2\sqrt{3}} \,+\, \frac{3}{2} \ln 3 \,-\, 3 \,.
\eeq
	Note that the difference of the superpotential values is proportional to the corresponding
	mass difference $ m_{k+1} \,-\, m_k $ with a {\it real} coefficient,
	at the logarithmic and finite orders
	(it is clear that the corrections will be complex, since at finite mass values,
	{\it i.e.} near the CMS the proportionality coefficient must become imaginary). 

	Equations \eqref{vval} agree with the previously derived formula, which in the current
	notations for the masses looks as 
\beq
\label{wk}
	\ww (\sigma_{k+1}) ~~=~~ e^{2\pi i /3 } \cdot \ww (\sigma_k) ~+~ 2\pi i \cdot m_2\,.
\eeq
	(with $ m_2 \,=\, |m|\, e^{i \pi /3} $).
	Again, if $ m $ is taken very large, then the logarithmic term in the superpotential
	dominates, and one has asymptotically,
\beq
	\ww (\sigma_{k+1}) ~~=~~ e^{2\pi i /3 } \cdot \ww (\sigma_k) \,, \qquad\qquad
				\lim~m~\gg~\Lambda\,.
\eeq

	Formula \eqref{wk} is valid for finite masses as well (as long as 
	$ |\sigma| \,<\, m $).
	The mass on the RHS of \eqref{wk} is independent of $ k $.
	Furthermore, the same formula was earlier derived for CP($N-1$) with odd $N$, for which 
	in general 
\beq
\label{wkN}
	\ww (\sigma_{k+1}) ~~=~~ e^{2\pi i /N } \cdot \ww (\sigma_k) ~+~ 2\pi i \cdot m_*\,.
\eeq
	where $ m_* $ is that mass which is $ |m|\,e^{i\pi/N} $.
	Then, one has
\beq
	\ww (\sigma_{p}) ~~=~~ e^{2\pi i p/N } \cdot \ww (\sigma_0)
		~+~ 2\pi i \bigl( m_* \,+\, m_{*+1} \,+\, \dots \,+\, m_{*+(p-1)} \bigr)\,.
\eeq
	Since the masses are on the circle, the bracket term forms a geometric progression,
	and therefore {\it is} expressible as the mass difference $ m_p \,-\, m_0 $ 
	(times a, in general, complex coefficient). 

	However, equations \eqref{vval}, \eqref{wk} and \eqref{wkN} were derived for
	{\it real} $ m_0 $.
	Generalization to complex $ m_0 $ needs to be done.



\begin{thebibliography}{99}
%\cite{Olmez:2007sg}
\bibitem{Olmez:2007sg}
  S.~Olmez and M.~Shifman,
  %``Curves of Marginal Stability in Two-Dimensional CP(N-1) Models with
  %Z_N-Symmetric Twisted Masses,''
  J.\ Phys.\ A  {\bf 40}, 11151 (2007)
  [arXiv:hep-th/0703149].
  %%CITATION = JPAGB,A40,11151;%%

%\cite{Dorey:1998yh}
\bibitem{Dorey:1998yh}
  N.~Dorey,
  %``The BPS spectra of two-dimensional supersymmetric gauge theories with
  %twisted mass terms,''
  JHEP {\bf 9811}, 005 (1998)
  [arXiv:hep-th/9806056].
  %%CITATION = JHEPA,9811,005;%%

%\cite{Hanany:1997vm}
\bibitem{Hanany:1997vm}
  A.~Hanany and K.~Hori,
  %``Branes and N = 2 theories in two dimensions,''
  Nucl.\ Phys.\  B {\bf 513}, 119 (1998)
  [arXiv:hep-th/9707192].
  %%CITATION = NUPHA,B513,119;%%

%\cite{Dorey:1999zk}
\bibitem{Dorey:1999zk}
  N.~Dorey, T.~J.~Hollowood and D.~Tong,
  %``The BPS spectra of gauge theories in two and four dimensions,''
  JHEP {\bf 9905}, 006 (1999)
  [arXiv:hep-th/9902134].
  %%CITATION = JHEPA,9905,006;%%

\end{thebibliography}



\end{document}

