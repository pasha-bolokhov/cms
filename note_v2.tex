%\documentclass{article}
\documentclass[12pt]{article}
\usepackage{latexsym}
\usepackage{amsmath}
\usepackage{amssymb}
\usepackage{relsize}
\usepackage{geometry}
\geometry{letterpaper}

%\usepackage{showlabels}

\textwidth = 6.0 in
\textheight = 8.5 in
\oddsidemargin = 0.0 in
\evensidemargin = 0.0 in
\topmargin = 0.2 in
\headheight = 0.0 in
\headsep = 0.0 in
%\parskip = 0.05in
\parindent = 0.35in


%% common definitions
\def\stackunder#1#2{\mathrel{\mathop{#2}\limits_{#1}}}
\def\beqn{\begin{eqnarray}}
\def\eeqn{\end{eqnarray}}
\def\nn{\nonumber}
\def\baselinestretch{1.1}
\def\beq{\begin{equation}}
\def\eeq{\end{equation}}
\def\ba{\beq\new\begin{array}{c}}
\def\ea{\end{array}\eeq}
\def\be{\ba}
\def\ee{\ea}
\def\stackreb#1#2{\mathrel{\mathop{#2}\limits_{#1}}}
\def\Tr{{\rm Tr}}
\newcommand{\gsim}{\lower.7ex\hbox{$
\;\stackrel{\textstyle>}{\sim}\;$}}
\newcommand{\lsim}{\lower.7ex\hbox{$
\;\stackrel{\textstyle<}{\sim}\;$}}
%%%%%%%%%%
\newcommand{\nfour}{${\mathcal N}=4$ }
\newcommand{\ntwo}{${\mathcal N}=2$ }
\newcommand{\ntwon}{${\mathcal N}=2$}
\newcommand{\ntwot}{${\mathcal N}= \left(2,2\right) $ }
\newcommand{\ntwoo}{${\mathcal N}= \left(0,2\right) $ }
\newcommand{\ntwoon}{${\mathcal N}= \left(0,2\right) $}
\newcommand{\none}{${\mathcal N}=1$ }
%%%%%%%%%%%
\newcommand{\nonen}{${\mathcal N}=1$}
\newcommand{\vp}{\varphi}
\newcommand{\pt}{\partial}
\newcommand{\ve}{\varepsilon}
\newcommand{\gs}{g^{2}}
\newcommand{\qt}{\tilde q}
%\renewcommand{\theequation}{\thesection.\arabic{equation}}

%%
\newcommand{\p}{\partial}
\newcommand{\wt}{\widetilde}
\newcommand{\ov}{\overline}
\newcommand{\mc}[1]{\mathcal{#1}}
\newcommand{\md}{\mathcal{D}}

\newcommand{\GeV}{{\rm GeV}}
\newcommand{\eV}{{\rm eV}}
\newcommand{\Heff}{{\mathcal{H}_{\rm eff}}}
\newcommand{\Leff}{{\mathcal{L}_{\rm eff}}}
\newcommand{\el}{{\rm EM}}
\newcommand{\uflavor}{\mathbf{1}_{\rm flavor}}
\newcommand{\lgr}{\left\lgroup}
\newcommand{\rgr}{\right\rgroup}

\newcommand{\Mpl}{M_{\rm Pl}}
\newcommand{\suc}{{{\rm SU}_{\rm C}(3)}}
\newcommand{\sul}{{{\rm SU}_{\rm L}(2)}}
\newcommand{\sutw}{{\rm SU}(2)}
\newcommand{\suth}{{\rm SU}(3)}
\newcommand{\ue}{{\rm U}(1)}

%%%%%%%%%%%%%%%%%%%%%%%%%%%%%%%%%%%%%%%
%  Slash character...
\def\slashed#1{\setbox0=\hbox{$#1$}             % set a box for #1
   \dimen0=\wd0                                 % and get its size
   \setbox1=\hbox{/} \dimen1=\wd1               % get size of /
   \ifdim\dimen0>\dimen1                        % #1 is bigger
      \rlap{\hbox to \dimen0{\hfil/\hfil}}      % so center / in box
      #1                                        % and print #1
   \else                                        % / is bigger
      \rlap{\hbox to \dimen1{\hfil$#1$\hfil}}   % so center #1
      /                                         % and print /
   \fi}                                         %
%%EXAMPLE:  $\slashed{E}$ or $\slashed{E}_{t}$

%%

\newcommand{\LN}{\Lambda_\text{SU($N$)}}
\newcommand{\sunu}{{\rm SU($N$) $\times$ U(1) }}
\newcommand{\sunun}{{\rm SU($N$) $\times$ U(1)}}
\def\cfl {$\text{SU($N$)}_{\rm C+F}$ }
\def\cfln {$\text{SU($N$)}_{\rm C+F}$}
\newcommand{\mUp}{m_{\rm U(1)}^{+}}
\newcommand{\mUm}{m_{\rm U(1)}^{-}}
\newcommand{\mNp}{m_\text{SU($N$)}^{+}}
\newcommand{\mNm}{m_\text{SU($N$)}^{-}}
\newcommand{\AU}{\mc{A}^{\rm U(1)}}
\newcommand{\AN}{\mc{A}^\text{SU($N$)}}
\newcommand{\aU}{a^{\rm U(1)}}
\newcommand{\aN}{a^\text{SU($N$)}}
\newcommand{\baU}{\ov{a}{}^{\rm U(1)}}
\newcommand{\baN}{\ov{a}{}^\text{SU($N$)}}
\newcommand{\lU}{\lambda^{\rm U(1)}}
\newcommand{\lN}{\lambda^\text{SU($N$)}}
%\newcommand{\Tr}{{\rm Tr\,}}
\newcommand{\bxir}{\ov{\xi}{}_R}
\newcommand{\bxil}{\ov{\xi}{}_L}
\newcommand{\xir}{\xi_R}
\newcommand{\xil}{\xi_L}
\newcommand{\bzl}{\ov{\zeta}{}_L}
\newcommand{\bzr}{\ov{\zeta}{}_R}
\newcommand{\zr}{\zeta_R}
\newcommand{\zl}{\zeta_L}
\newcommand{\nbar}{\ov{n}}

\newcommand{\ssm}{{\scriptscriptstyle(M)}}
\newcommand{\sse}{{\scriptscriptstyle(E)}}
\newcommand{\cell}{{\mathcal L}}
\newcommand{\CPC}{CP($N-1$)$\times$C }
\newcommand{\CPCn}{CP($N-1$)$\times$C}
\newcommand{\cpn}{CP$(N-1)\,$}

\newcommand{\lar}{\lambda_R}
\newcommand{\lal}{\lambda_L}
\newcommand{\larl}{\lambda_{R,L}}
\newcommand{\lalr}{\lambda_{L,R}}
\newcommand{\blar}{\ov{\lambda}{}_R}
\newcommand{\blal}{\ov{\lambda}{}_L}
\newcommand{\blarl}{\ov{\lambda}{}_{R,L}}
\newcommand{\blalr}{\ov{\lambda}{}_{L,R}}

\newcommand{\tgamma}{\wt{\gamma}}
\newcommand{\btgamma}{\ov{\tgamma}}
\newcommand{\bpsi}{\ov{\psi}{}}
\newcommand{\bphi}{\ov{\phi}{}}
\newcommand{\bxi}{\ov{\xi}{}}

\newcommand{\ff}{\mc{F}}
\newcommand{\bff}{\ov{\mc{F}}}

\newcommand{\eer}{\epsilon_R}
\newcommand{\eel}{\epsilon_L}
\newcommand{\eerl}{\epsilon_{R,L}}
\newcommand{\eelr}{\epsilon_{L,R}}
\newcommand{\beer}{\ov{\epsilon}{}_R}
\newcommand{\beel}{\ov{\epsilon}{}_L}
\newcommand{\beerl}{\ov{\epsilon}{}_{R,L}}
\newcommand{\beelr}{\ov{\epsilon}{}_{L,R}}

\newcommand{\bi}{{\bar \imath}}
\newcommand{\bj}{{\bar \jmath}}
\newcommand{\bk}{{\bar k}}
\newcommand{\bl}{{\bar l}}
\newcommand{\bm}{{\bar m}}

\newcommand{\hsigma}{{\hat{\sigma}}}

\begin{document}

{\centering\bf
\large\underline{A note on CMS curve in supersymmetric CP($N-1$) model}}

%\vspace{1.5cm}
\section*{Remark on Argyres-Douglas points}
	If one considers a generic CP($N-1$) with twisted masses $m_j$, the latter would not have to be
situated on a circle, although, of course, that would be helpful for calculations. 
	If the situation with collision of a few vacua is to be investigated, one effectively arrives
at a theory CP($n-1$) where $n$ is the number of colliding vacua, two or more. 
	In such an effective description, now all of the vacua of CP($n-1$) are colliding. 
We then just assume that $ n = N $.

	If all $N$ vacua collide, then there is no other way than all corresponding masses
have to sit on the circle. 
	Whether the masses have not been distributed on a $Z_N$ circle initially, they will
become so. For now we adopt that the masses were on the circle at all times.

\section*{Massive CP(1)}

	The simplest example to consider is CP(1). 
The two masses $m_j$ are of the opposite sign, and so are the vacua $\sigma_j$.
If one starts from the effective superpotential
\beq
\label{Wexact}
	-2\, \mc{W}_\text{eff} (\hsigma) ~~=~~ i\, \tau\hsigma ~-~ 
		\frac{1}{2\pi} \sum_j (\hsigma - m_j)\, 
				      \left\{ \ln {\frac{\hsigma - m_j}{\mu}} ~-~ 1 \right\}\,,
\eeq
one finds for the vacuum values, 
\beq
	\mc{W}_\text{eff} (\sigma_p) ~~=~~ -\, \frac{N}{4\pi}\,\sigma_p ~-~
				\frac{1}{4\pi} \sum_j m_j\, \ln \frac{\sigma_p - m_j}{\mu}\,,
\eeq
or, since the sum of the masses can be set to zero,
\beq
\label{vvalue}
	\mc{W}_\text{eff} (\sigma_p) ~~=~~ -\, \frac{N}{4\pi}\,\sigma_p ~-~
				\frac{1}{4\pi} \sum_j m_j\, \ln ( \sigma_p - m_j ) \,.
\eeq
	Superpotential \eqref{Wexact} can be rewritten as
\beq
	\mc{W}_\text{eff} (\hsigma) ~~=~~-\, \frac{1}{2}i \left(\frac{\theta}{2\pi}\right)\, \hsigma ~+~
		\frac{1}{4\pi} \sum_j (\hsigma - m_j)\, 
					\left\{ \ln \frac{\hsigma - m_j}{\Lambda} ~-~ 1 \right\}\,,
\eeq
	via
\beq
	\mu ~~=~~ \Lambda\, e^{2\pi r/N}   
	\qquad\qquad \text{and} \qquad\qquad
	\tau ~~=~~ i r ~+~ \frac{\theta}{2\pi}\,.
\eeq
	We, of course, will set $ \theta $ angle to zero, and so
\beq
\label{Weff}
	\mc{W}_\text{eff} (\hsigma) ~~=~~ \frac{1}{4\pi} \sum_j (\hsigma - m_j)\, 
					  \left\{ \ln \frac{\hsigma - m_j}{\Lambda} ~-~ 1 \right\}\,.
\eeq

	This has been valid for arbitrary $N$, let us now take $N$ equal to 2.
	Consider now the relative position of the vacuum values \eqref{vvalue} in CP(1). 
	Take mass $ m_1 $ large and positive, and then $ m_{-1} $ will be large and negative. 
	The vacua $ \sigma_1 $ and $ \sigma_{-1} $ are also large. 
	The vacuum values are
\begin{align}
%
\notag
	\mc{W}(\sigma_1) & ~~=~~ \frac{m}{2\pi}\,
			\Biggl\lgroup \ln\, \frac{2m}{\Lambda}  ~-~ 1 ~-~ \Big(\frac{\Lambda}{2m}\Big)^2 
			              ~+~ ... \Biggr\rgroup ,
	\\
%
\label{positivem}
	\mc{W}(\sigma_{-1}) & ~~=~~ - \mc{W}(\sigma_1) ~-~
			i\pi\, \frac{m}{2\pi}\, 
			\Biggl\lgroup 1 ~+~ 2\,\Big( \frac{\Lambda}{2m} \Big)^2 \Biggr\rgroup .
\end{align}
	The last term in the second line of \eqref{positivem} prevents the vacuum values to be of the
	opposite sign.
	However, quite generally, the logarithms in Eq.~\eqref{Weff} could have additions 
	proportional to integer numbers of $ 2\pi i $.
	Let us, in a very general setting, account for these ambiguities.
	Then, ignoring $(\Lambda/2m)^2$ corrections, it is possible 
	to cancel the last term on the second line of Eq.~\eqref{positivem}, by the appropriate choice of 
	integer numbers times $ 2\pi i\, m_j $.
	We could {\it define} the vacuum value $\mc{W}(\sigma_{-1})$ such that it is the exact opposite
	of $\mc{W}(\sigma_1)$.
	But, again, in expression \eqref{positivem} it is not such.

	
	Now let us approach the two-dimensional Argyres-Douglas point, which in this case is 
$ m_j \,=\, \{\, -i \Lambda,\, +i \Lambda \,\} $. 
	For the moment we ignore the issue of branching of the logarithms, as we are currently interested
	in the absolute value of $ \mc{W}(\sigma_p) $.
	Then we can just set $ \sigma \to 0 $, and, substituting the masses into expression \eqref{vvalue},
	we obtain,
\beq
\label{vaccp1}
	\mc{W}_\text{eff}(\sigma_1) ~~=~~ -\, \frac{1}{4\pi}\, \sum_j m_j\, \ln(-m_j)  ~~=~~ 
			-\, \frac{1}{4\pi}\cdot \pi\, \Lambda \,.
\eeq
	One can formally calculate $ \mc{W}_\text{eff}(\sigma_{-1}) $ by pulling out the factor of $ e^{2\pi i / 2} $ from
	the logarithms in Eq.~\eqref{vvalue}, and this way expressing the value in terms of $ \mc{W}_\text{eff}(\sigma_1) $.
	One still has to put $ \sigma $ to zero after this process, and, by correctly taking into account the phases of the
	logarithms now,
\beq
\label{relvalcp1}
	\mc{W}_\text{eff}(\sigma_{-1}) ~~=~~ -\, \mc{W}_\text{eff}(\sigma_1) ~~+~~ \frac{1}{2}\, \Lambda \,,
\eeq	
	which is consistent with \eqref{vaccp1} if we set $ \mc{W}_\text{eff}(\sigma_1) = \mc{W}_\text{eff}(\sigma_{-1}) $.
	Then one has massless states in the Argyres-Douglas point, and the latter lies on the CMS. 

	Expressions \eqref{vaccp1} and \eqref{relvalcp1} are not consistent with the superpotential values to be of the opposite sign.
	For, if they were of the opposite sign, their values would have to vanish in order for the model to have
	massless states at the Argyres-Douglas point. 
	The $ Z_N $ property, however, can be enforced by choosing the appropriate branches of the logarithms in
	\eqref{vvalue}. 
	Then one can set $ \mc{W}_\text{eff}(\sigma_1) = - \mc{W}_\text{eff}(\sigma_{-1}) = 0 $, and there are massless states.

\section*{Massless CP({\boldmath $N-1$})}

	In the massless theory, $Z_N$ property is evident from \eqref{vvalue},
\beq
	\mc{W}(\sigma_p) ~~=~~ -\, \frac{N}{4\pi}\, \sigma_p \,.
\eeq
	The vacua are found from minimization of the potential,
\beq
\label{Vmassless}
	V ~~\propto~~ \left|\, \wt{\tau} ~-~ \frac{N}{2\pi i}\, \ln\, \sigma/\mu \,\right|^2\,.
\eeq
	Here 
\beq
\label{twtau}
	\wt{\tau} ~~=~~ \tau ~+~ n^*\,,
\eeq	
	where $ n^* \in {\mc Z} $ is found from minimizing the absolute value of the effective $\theta$-angle
\beq	
	\left|\, \frac{\theta}{2\pi} ~-~ \frac{N}{2\pi}\, {\rm Arg}\, \sigma ~+~ n^* \,\right| \,.
\eeq
	Then (if we set $\theta = 0$) the vacua are
\beq
	\sigma_k ~~=~~ \Lambda\, e^{2\pi i k / N}\,,
\eeq
	as for each value of $k$ there will be the corresponding $ n^* ~=~ k $.
	The RHS of \eqref{Vmassless} can be written as
\beq
	\left|\, \frac{\wt{\theta}}{2\pi} ~-~ \frac{N}{2\pi i}\, \ln\, \sigma/\Lambda   \,\right|^2 \,,
\eeq
	with $ \wt{\theta} $ from $ \eqref{twtau} $, and obviously vanishes in the vacua.

\section*{CP(1) in the Mirror Representation}

	We can look at what superpotential values look like in the mirror representation.
	One has,
\beq
	\mc{W}_\text{mirror}^\text{CP(1)} ~~=~~
		-\, \frac{\Lambda}{4\pi} 
			\lgr  ( X + \frac{1}{X} ) ~-~
			      \left\{ \frac{m_1}{\Lambda}\, \ln\, X ~+~
				      \frac{m_2}{\Lambda}\, \ln\, \frac{1}{X} \right\} \rgr.
\eeq
	Let us take masses $ m_1 $ and $ m_2 $ of the opposite sign, and approach the CMS.
	Denoting with $ X^{(1)} $ and $ X^{(2)} $ the possible vacua, for the vacuum values we find
\begin{align}
%
\label{mirrvv}
	\mc{W}_\text{mirror}^\text{CP(1)}\bigg|_\text{vacuum (1)} & ~~=~~
		-\, \frac{\Lambda}{4\pi} 
			\lgr \sqrt{ (\Delta m / \Lambda)^2 + 4\, } ~~-~~ \frac{\Delta m}{\Lambda}\, \ln\, X^{(1)} \rgr,
	\\
%
\notag
	\mc{W}_\text{mirror}^\text{CP(1)}\bigg|_\text{vacuum (2)} & ~~=~~
		-\, \frac{\Lambda}{4\pi}
			\lgr -\, \sqrt{ (\Delta m / \Lambda)^2 + 4\, } ~~+~~ 
			 \frac{\Delta m}{\Lambda}\, \big(\ln\, X^{(1)} ~-~ i\pi \big) \rgr ,
\end{align}
	with
\beq
\label{mirrv}
	X^{(1,2)} ~~=~~  \frac{\,  \Delta m / \Lambda \,\pm\, \sqrt{ (\Delta m / \Lambda)^2 + 4\, } \,}
                                                      {  2  }\,.
\eeq
	Equations \eqref{mirrvv} and \eqref{mirrv} are valid for any masses --- small, opposite sign or not.
	As long as they are inside the CMS, one can compare the vacuum values in \eqref{mirrvv}.
	In particular, one can take $ \Delta m \,\sim\, \Lambda $ to approach the CMS,
	then $ X^{(1,2)} \,\sim\, 1 $ and so are $ \ln\, X^{(1)} $ and $ \ln\, X^{(2)} $.


\begin{thebibliography}{99}
%\cite{Olmez:2007sg}
\bibitem{Olmez:2007sg}
  S.~Olmez and M.~Shifman,
  %``Curves of Marginal Stability in Two-Dimensional CP(N-1) Models with
  %Z_N-Symmetric Twisted Masses,''
  J.\ Phys.\ A  {\bf 40}, 11151 (2007)
  [arXiv:hep-th/0703149].
  %%CITATION = JPAGB,A40,11151;%%

%\cite{Dorey:1998yh}
\bibitem{Dorey:1998yh}
  N.~Dorey,
  %``The BPS spectra of two-dimensional supersymmetric gauge theories with
  %twisted mass terms,''
  JHEP {\bf 9811}, 005 (1998)
  [arXiv:hep-th/9806056].
  %%CITATION = JHEPA,9811,005;%%

%\cite{Hanany:1997vm}
\bibitem{Hanany:1997vm}
  A.~Hanany and K.~Hori,
  %``Branes and N = 2 theories in two dimensions,''
  Nucl.\ Phys.\  B {\bf 513}, 119 (1998)
  [arXiv:hep-th/9707192].
  %%CITATION = NUPHA,B513,119;%%

\end{thebibliography}



\end{document}

