%\documentclass{article}
\documentclass[12pt]{article}
\usepackage{latexsym}
\usepackage{amsmath}
\usepackage{amssymb}
\usepackage{relsize}
\usepackage{geometry}
\geometry{letterpaper}

\usepackage{showlabels}

\textwidth = 6.0 in
\textheight = 8.5 in
\oddsidemargin = 0.0 in
\evensidemargin = 0.0 in
\topmargin = 0.2 in
\headheight = 0.0 in
\headsep = 0.0 in
%\parskip = 0.05in
\parindent = 0.35in


%% common definitions
\def\stackunder#1#2{\mathrel{\mathop{#2}\limits_{#1}}}
\def\beqn{\begin{eqnarray}}
\def\eeqn{\end{eqnarray}}
\def\nn{\nonumber}
\def\baselinestretch{1.1}
\def\beq{\begin{equation}}
\def\eeq{\end{equation}}
\def\ba{\beq\new\begin{array}{c}}
\def\ea{\end{array}\eeq}
\def\be{\ba}
\def\ee{\ea}
\def\stackreb#1#2{\mathrel{\mathop{#2}\limits_{#1}}}
\def\Tr{{\rm Tr}}
\newcommand{\gsim}{\lower.7ex\hbox{$
\;\stackrel{\textstyle>}{\sim}\;$}}
\newcommand{\lsim}{\lower.7ex\hbox{$
\;\stackrel{\textstyle<}{\sim}\;$}}
%%%%%%%%%%
\newcommand{\nfour}{${\mathcal N}=4$ }
\newcommand{\ntwo}{${\mathcal N}=2$ }
\newcommand{\ntwon}{${\mathcal N}=2$}
\newcommand{\ntwot}{${\mathcal N}= \left(2,2\right) $ }
\newcommand{\ntwoo}{${\mathcal N}= \left(0,2\right) $ }
\newcommand{\ntwoon}{${\mathcal N}= \left(0,2\right) $}
\newcommand{\none}{${\mathcal N}=1$ }
%%%%%%%%%%%
\newcommand{\nonen}{${\mathcal N}=1$}
\newcommand{\vp}{\varphi}
\newcommand{\pt}{\partial}
\newcommand{\ve}{\varepsilon}
\newcommand{\gs}{g^{2}}
\newcommand{\qt}{\tilde q}
%\renewcommand{\theequation}{\thesection.\arabic{equation}}

%%
\newcommand{\p}{\partial}
\newcommand{\wt}{\widetilde}
\newcommand{\ov}{\overline}
\newcommand{\mc}[1]{\mathcal{#1}}
\newcommand{\md}{\mathcal{D}}

\newcommand{\GeV}{{\rm GeV}}
\newcommand{\eV}{{\rm eV}}
\newcommand{\Heff}{{\mathcal{H}_{\rm eff}}}
\newcommand{\Leff}{{\mathcal{L}_{\rm eff}}}
\newcommand{\el}{{\rm EM}}
\newcommand{\uflavor}{\mathbf{1}_{\rm flavor}}
\newcommand{\lgr}{\left\lgroup}
\newcommand{\rgr}{\right\rgroup}

\newcommand{\Mpl}{M_{\rm Pl}}
\newcommand{\suc}{{{\rm SU}_{\rm C}(3)}}
\newcommand{\sul}{{{\rm SU}_{\rm L}(2)}}
\newcommand{\sutw}{{\rm SU}(2)}
\newcommand{\suth}{{\rm SU}(3)}
\newcommand{\ue}{{\rm U}(1)}
%%%%%%%%%%%%%%%%%%%%%%%%%%%%%%%%%%%%%%%
%  Slash character...
\def\slashed#1{\setbox0=\hbox{$#1$}             % set a box for #1
   \dimen0=\wd0                                 % and get its size
   \setbox1=\hbox{/} \dimen1=\wd1               % get size of /
   \ifdim\dimen0>\dimen1                        % #1 is bigger
      \rlap{\hbox to \dimen0{\hfil/\hfil}}      % so center / in box
      #1                                        % and print #1
   \else                                        % / is bigger
      \rlap{\hbox to \dimen1{\hfil$#1$\hfil}}   % so center #1
      /                                         % and print /
   \fi}                                        %

%%EXAMPLE:  $\slashed{E}$ or $\slashed{E}_{t}$

%%

\newcommand{\LN}{\Lambda_\text{SU($N$)}}
\newcommand{\sunu}{{\rm SU($N$) $\times$ U(1) }}
\newcommand{\sunun}{{\rm SU($N$) $\times$ U(1)}}
\def\cfl {$\text{SU($N$)}_{\rm C+F}$ }
\def\cfln {$\text{SU($N$)}_{\rm C+F}$}
\newcommand{\mUp}{m_{\rm U(1)}^{+}}
\newcommand{\mUm}{m_{\rm U(1)}^{-}}
\newcommand{\mNp}{m_\text{SU($N$)}^{+}}
\newcommand{\mNm}{m_\text{SU($N$)}^{-}}
\newcommand{\AU}{\mc{A}^{\rm U(1)}}
\newcommand{\AN}{\mc{A}^\text{SU($N$)}}
\newcommand{\aU}{a^{\rm U(1)}}
\newcommand{\aN}{a^\text{SU($N$)}}
\newcommand{\baU}{\ov{a}{}^{\rm U(1)}}
\newcommand{\baN}{\ov{a}{}^\text{SU($N$)}}
\newcommand{\lU}{\lambda^{\rm U(1)}}
\newcommand{\lN}{\lambda^\text{SU($N$)}}
%\newcommand{\Tr}{{\rm Tr\,}}
\newcommand{\bxir}{\ov{\xi}{}_R}
\newcommand{\bxil}{\ov{\xi}{}_L}
\newcommand{\xir}{\xi_R}
\newcommand{\xil}{\xi_L}
\newcommand{\bzl}{\ov{\zeta}{}_L}
\newcommand{\bzr}{\ov{\zeta}{}_R}
\newcommand{\zr}{\zeta_R}
\newcommand{\zl}{\zeta_L}
\newcommand{\nbar}{\ov{n}}

\newcommand{\ssm}{{\scriptscriptstyle(M)}}
\newcommand{\sse}{{\scriptscriptstyle(E)}}
\newcommand{\cell}{{\mathcal L}}
\newcommand{\CPC}{CP($N-1$)$\times$C }
\newcommand{\CPCn}{CP($N-1$)$\times$C}
\newcommand{\cpn}{CP$(N-1)\,$}

\newcommand{\lar}{\lambda_R}
\newcommand{\lal}{\lambda_L}
\newcommand{\larl}{\lambda_{R,L}}
\newcommand{\lalr}{\lambda_{L,R}}
\newcommand{\blar}{\ov{\lambda}{}_R}
\newcommand{\blal}{\ov{\lambda}{}_L}
\newcommand{\blarl}{\ov{\lambda}{}_{R,L}}
\newcommand{\blalr}{\ov{\lambda}{}_{L,R}}

\newcommand{\tgamma}{\wt{\gamma}}
\newcommand{\btgamma}{\ov{\tgamma}}
\newcommand{\bpsi}{\ov{\psi}{}}
\newcommand{\bphi}{\ov{\phi}{}}
\newcommand{\bxi}{\ov{\xi}{}}

\newcommand{\ff}{\mc{F}}
\newcommand{\bff}{\ov{\mc{F}}}

\newcommand{\eer}{\epsilon_R}
\newcommand{\eel}{\epsilon_L}
\newcommand{\eerl}{\epsilon_{R,L}}
\newcommand{\eelr}{\epsilon_{L,R}}
\newcommand{\beer}{\ov{\epsilon}{}_R}
\newcommand{\beel}{\ov{\epsilon}{}_L}
\newcommand{\beerl}{\ov{\epsilon}{}_{R,L}}
\newcommand{\beelr}{\ov{\epsilon}{}_{L,R}}

\newcommand{\bi}{{\bar \imath}}
\newcommand{\bj}{{\bar \jmath}}
\newcommand{\bk}{{\bar k}}
\newcommand{\bl}{{\bar l}}
\newcommand{\bm}{{\bar m}}

\begin{document}

{\centering
\large\underline{A note on AD point vs. CMS curve in supersymmetric CP($N-1$) model}}

\vspace{1.5cm}
The effective superpotential in the mirror representation reads \cite{Olmez:2007sg,Dorey:1998yh}
\beq
\label{Wgen}
	\mc{W}(\sigma) ~~=~~ \frac{N}{4\pi}  
		\lgr  \sigma ~-~ \frac{1}{N}\, \sum m_j\, \ln ( \sigma \;+\; m_j ) \rgr .
\eeq
Its vacua are located at
\beq
\label{sigmap}
	\sigma_{(p)} ~~=~~ \sqrt[N]{ 1 \,+\,  (- m_0)^N }\, e^{2\pi i p / N}\,.
\eeq

Both the vacua $ \sigma_{(p)} $ and the masses $ m_j $ satisfy the $ Z_N $ symmetry. 
The Argyres-Douglas point is, therefore when all $ \sigma $'s vanish:
\beq
	\sigma_0  ~~=~~ \sigma_{(p)} ~~=~~ \dots ~~=~~ 0\, \qquad\qquad \forall~ p.
\eeq

The CMS condition is the equality of the phases of the fundamental and topological masses,
\beq
	Z  ~~=~~ -i \lgr   (m_k ~-~ m_l )  ~~+~~ (m_D^{(k)} ~-~ m_D^{(l)}) \rgr,
\eeq
where the latter are
\beq
	m_D^{(k)} ~~=~~ 2i\, \mc{W}(\sigma_{(k)})\,,
\eeq
and, therefore
\beq
\label{Z}
	Z  ~~=~~ -i \lgr (m_k ~-~ m_l )  ~~+~~ 2i\, \left\{ \mc{W}(\sigma_{(k)}) ~-~ \mc{W}(\sigma_{(l)}) \right\} \rgr.
\eeq

It seems that, as $ \sigma \to 0 $, the topological part of \eqref{Z} goes to zero and the equality of
phases --- {\it i.e.} the CMS condition --- is satisfied.

However, equation \eqref{Wgen} for $\mc{W}(\sigma)$ does not mean that literally
\[
	\mc{W}(\sigma_{(p)}) ~~=~~ \frac{N}{4\pi}  
		\lgr  \sigma_{(p)} ~-~ \frac{1}{N}\, \sum m_j\, \ln ( \sigma_{(p)} \;+\; m_j ) \rgr ,
\]
as in general, each logarithm has a cut and admits an addition of $ 2\pi i $ times an integer. 
The latter integer constants are fixed by imposing the $ Z_N $ symmetry on the vacuum values of the superpotential,
\beq
\label{WZN}
	\mc{W}(\sigma_{(p)}) ~~=~~ e^{2\pi i p/N}\, \mc{W}(\sigma_0) \,,
\eeq
{\it i.e.} the vacuum values also sit on the circle. 
The vacuum values therefore could only cancel in \eqref{Z} if this circle shrinks to zero, which does not
happen. 

The value $ \mc{W}(\sigma_{(p)}) $ can be related to the value in the zeroth vacuum 
$ \mc{W}(\sigma_0) $ as in \cite{Olmez:2007sg}, by substituting 
\[
	\sigma_{(p)} ~~=~~ e^{2\pi i p/N}\, \sigma_0\,.
\]
However, analysis of branch cuts of the logarithms shows that in this case
\[
	\mc{W}(\sigma_{(p)})  ~~\neq~~   e^{2\pi i p/N}\, \mc{W}(\sigma_0)
\]
if $ \mc{W}(\sigma) $ is just taken literally from Eq.~\eqref{Wgen}. The above expression is understood
to be taken in the limit $ |\sigma_0| \to 0 $, {\it i.e.} near the AD point.

The limit of the AD point is useful, since then one can correctly neglect the phases of the $ \sigma $'s in
the logarithms  ($ \arg (\sigma + m_j) ~\approx~ \arg m_j $), and analyse which of the logarithms
may exceed their branch cuts. Imposition of $Z_N$ symmetry then demands to cancel the corresponding
constants that pop out. 
The result is, 
\begin{align}
%
\notag
	\text{true}~ \mc{W}(\sigma_{(p)})  & ~~=~~ 
		\frac{N}{4\pi}  
		\lgr  \sigma_{(p)} ~-~ \frac{1}{N}\, \sum m_j\, \ln ( \sigma_{(p)} \;+\; m_j ) \rgr 
	~+~ \\
%
\label{Wtrue}
	&
\qquad
	~~+~~
	\frac{i}{2} \left\{
	\begin{array}{ll}
		  {\displaystyle \sum_{ j > n - |p|}^{n}}\,  ( -m_j)\,, &  p < 0 \\
			\\
		  {\displaystyle \sum_{j=-n}^{j<-n + p}}\,   m_j\,, &  p > 0
	\end{array}
	\right.,
	\\
%
\notag
	& [\lim~ \sigma_0~ \to~ 0]\,,  \\
%
\notag
	& [\text{for CP($N-1$) with {\it odd} $N$}].
\end{align}
The above equation is meant to apply in the limit where masses $ m_j $ approach the AD point $ \sigma_0 = 0 $
(therefore, one can simply throw away the $ \sigma $'s on the RHS of \eqref{Wtrue}).
This equation is derived for the case of odd $N$.
On one hand, Eq.~\eqref{Wtrue} restores the $ Z_N $ symmetry, on the other hand, it allows to efficiently
compare the vacuum values of the superpotential (at least in the AD point).

The conventions for the masses and vacua in Eq.~\eqref{Wtrue} are as follows: masses are numbered as
$ -(N-1)/2$, ..., $0$, ..., $+(N-1)/2 $, with the same numbering for the vacua $ \sigma_{(p)} $. Therefore,
$ m_0 $ and $ \sigma_0 $ are real and positive. The cuts of each logarithm in \eqref{Wgen} are assumed to
be directed in the real negative direction, with the phase of the argument varying from 
$ - i \pi $ to $ + i \pi $.


Equation \eqref{Wtrue} means that for the $p$-th vacuum, $ p $ masses have to be added 
(or subtracted, depending on the sign of $p$) to the initial expression \eqref{Wgen}. 
Again, it was derived for the case of odd $N$, when both $ m_0 $ and $ \sigma_0 $ can be made real.
[But the confidence is that a very similar result holds for {\it even} $N$ as well --- $ p $
masses have to be added. The following discussion was not peformed for even $N$, however].

Using Eq.~\eqref{Wtrue} it is easy to show that for theory with odd $N$ the phase of 
$ m_D^{(k)} - m_D^{(l)} $ in the limit $ \sigma_0 \to 0 $ always differs from the 
phase of $ m^k - m^l $ by a factor of $ i $, and therefore they can never be made equal. 
Therefore, the AD point does not lie on the CMS.

On the other hand, for CP(1) theory ($N = 2$), this does not happen. 
The masses $m_j$ are imaginary at the AD point (see Eq.~\eqref{sigmap}), 
\[
	m_{(1)}  ~~\to~~ i\,, \qquad\qquad  m_{(-1)}  ~~\to~~  -i\,,
\]
while the superpotential is real (the sum of imaginary masses times their logarithms),
\[
	\mc{W}(\sigma_{(1)})  ~~=~~ - \mc{W}(\sigma_{(-1)}) ~~\in~~ \mc{R}\,.
\]
The extra factor of $ 2i $ in Eq.~\eqref{Z} therefore balances the phases, and the AD point
is on the CMS. 
What happens in general CP($N-1$) with even $N = 2 m$ remains to be seen.


\begin{thebibliography}{99}
%\cite{Olmez:2007sg}
\bibitem{Olmez:2007sg}
  S.~Olmez and M.~Shifman,
  %``Curves of Marginal Stability in Two-Dimensional CP(N-1) Models with
  %Z_N-Symmetric Twisted Masses,''
  J.\ Phys.\ A  {\bf 40}, 11151 (2007)
  [arXiv:hep-th/0703149].
  %%CITATION = JPAGB,A40,11151;%%

%\cite{Dorey:1998yh}
\bibitem{Dorey:1998yh}
  N.~Dorey,
  %``The BPS spectra of two-dimensional supersymmetric gauge theories with
  %twisted mass terms,''
  JHEP {\bf 9811}, 005 (1998)
  [arXiv:hep-th/9806056].
  %%CITATION = JHEPA,9811,005;%%

\end{thebibliography}



\end{document}

