\documentclass[epsfig,12pt]{article}
\usepackage{epsfig}
\usepackage{graphicx}
\usepackage{rotating}

%%%%%%%%%
\usepackage{latexsym}
\usepackage{amsmath}
\usepackage{amssymb}
\usepackage{relsize}
\usepackage{geometry}
\geometry{letterpaper}
\usepackage{color}
\usepackage{bm}
\usepackage{showlabels}
%%%%%%%%%%%%

\def\beq{\begin{equation}}
\def\eeq{\end{equation}}
\def\beqn{\begin{eqnarray}}
\def\eeqn{\end{eqnarray}}
\def\Tr{{\rm Tr}}
\newcommand{\nfour}{${\cal N}=4\;$}
\newcommand{\ntwo}{${\mathcal N}=2\,$}
\newcommand{\none}{${\mathcal N}=1\,$}
\newcommand{\ntt}{${\mathcal N}=(2,2)\,$}
\newcommand{\nzt}{${\mathcal N}=(0,2)\,$}
\newcommand{\cpn}{CP$(N-1)\,$}
\newcommand{\ca}{{\mathcal A}}
\newcommand{\cell}{{\mathcal L}}
\newcommand{\cw}{{\mathcal W}}
\newcommand{\cs}{{\mathcal S}}
\newcommand{\vp}{\varphi}
\newcommand{\pt}{\partial}
\newcommand{\ve}{\varepsilon}
\newcommand{\gs}{g^{2}}
\newcommand{\zn}{$Z_N$}
\newcommand{\cd}{${\mathcal D}$}
\newcommand{\cde}{{\mathcal D}}
\newcommand{\cf}{${\mathcal F}$}
\newcommand{\cfe}{{\mathcal F}}

\newcommand{\gsim}{\lower.7ex\hbox{$
\;\stackrel{\textstyle>}{\sim}\;$}}
\newcommand{\lsim}{\lower.7ex\hbox{$
\;\stackrel{\textstyle<}{\sim}\;$}}

\renewcommand{\theequation}{\thesection.\arabic{equation}}

%%%%%%%%%%%%%
%%%%%%%%%%%
%% common definitions
\def\stackunder#1#2{\mathrel{\mathop{#2}\limits_{#1}}}
\def\beqn{\begin{eqnarray}}
\def\eeqn{\end{eqnarray}}
\def\nn{\nonumber}
\def\baselinestretch{1.1}
\def\beq{\begin{equation}}
\def\eeq{\end{equation}}
\def\ba{\beq\new\begin{array}{c}}
\def\ea{\end{array}\eeq}
\def\be{\ba}
\def\ee{\ea}
\def\stackreb#1#2{\mathrel{\mathop{#2}\limits_{#1}}}
\def\Tr{{\rm Tr}}
%\newcommand{\gsim}{\lower.7ex\hbox{$\;\stackrel{\textstyle>}{\sim}\;$}}
% \newcommand{\lsim}{\lower.7ex\hbox{$
%\;\stackrel{\textstyle<}{\sim}\;$}}
%\newcommand{\nfour}{${\mathcal N}=4$ }
%\newcommand{\ntwo}{${\mathcal N}=2$ }
\newcommand{\ntwon}{${\mathcal N}=2$}
\newcommand{\ntwot}{${\mathcal N}= \left(2,2\right) $ }
\newcommand{\ntwoo}{${\mathcal N}= \left(0,2\right) $ }
%\newcommand{\none}{${\mathcal N}=1$ }
\newcommand{\nonen}{${\mathcal N}=1$}
%\newcommand{\vp}{\varphi}
%\newcommand{\pt}{\partial}
%\newcommand{\ve}{\varepsilon}
%\newcommand{\gs}{g^{2}}
%\newcommand{\qt}{\tilde q}
\renewcommand{\theequation}{\thesection.\arabic{equation}}

%%
\newcommand{\p}{\partial}
\newcommand{\wt}{\widetilde}
\newcommand{\ov}{\overline}
\newcommand{\mc}[1]{\mathcal{#1}}
\newcommand{\md}{\mathcal{D}}

\newcommand{\GeV}{{\rm GeV}}
\newcommand{\eV}{{\rm eV}}
\newcommand{\Heff}{{\mathcal{H}_{\rm eff}}}
\newcommand{\Leff}{{\mathcal{L}_{\rm eff}}}
\newcommand{\el}{{\rm EM}}
\newcommand{\uflavor}{\mathbf{1}_{\rm flavor}}
\newcommand{\lgr}{\left\lgroup}
\newcommand{\rgr}{\right\rgroup}

\newcommand{\Mpl}{M_{\rm Pl}}
\newcommand{\suc}{{{\rm SU}_{\rm C}(3)}}
\newcommand{\sul}{{{\rm SU}_{\rm L}(2)}}
\newcommand{\sutw}{{\rm SU}(2)}
\newcommand{\suth}{{\rm SU}(3)}
\newcommand{\ue}{{\rm U}(1)}
%%%%%%%%%%%%%%%%%%%%%%%%%%%%%%%%%%%%%%%
%  Slash character...
\def\slashed#1{\setbox0=\hbox{$#1$}             % set a box for #1
   \dimen0=\wd0                                 % and get its size
   \setbox1=\hbox{/} \dimen1=\wd1               % get size of /
   \ifdim\dimen0>\dimen1                        % #1 is bigger
      \rlap{\hbox to \dimen0{\hfil/\hfil}}      % so center / in box
      #1                                        % and print #1
   \else                                        % / is bigger
      \rlap{\hbox to \dimen1{\hfil$#1$\hfil}}   % so center #1
      /                                         % and print /
   \fi}                                        %

%%EXAMPLE:  $\slashed{E}$ or $\slashed{E}_{t}$

%%

\newcommand{\LN}{\Lambda_\text{SU($N$)}}
\newcommand{\sunu}{{\rm SU($N$) $\times$ U(1) }}
\newcommand{\sunun}{{\rm SU($N$) $\times$ U(1)}}
\def\cfl {$\text{SU($N$)}_{\rm C+F}$ }
\def\cfln {$\text{SU($N$)}_{\rm C+F}$}
\newcommand{\mUp}{m_{\rm U(1)}^{+}}
\newcommand{\mUm}{m_{\rm U(1)}^{-}}
\newcommand{\mNp}{m_\text{SU($N$)}^{+}}
\newcommand{\mNm}{m_\text{SU($N$)}^{-}}
\newcommand{\AU}{\mc{A}^{\rm U(1)}}
\newcommand{\AN}{\mc{A}^\text{SU($N$)}}
\newcommand{\aU}{a^{\rm U(1)}}
\newcommand{\aN}{a^\text{SU($N$)}}
\newcommand{\baU}{\ov{a}{}^{\rm U(1)}}
\newcommand{\baN}{\ov{a}{}^\text{SU($N$)}}
\newcommand{\lU}{\lambda^{\rm U(1)}}
\newcommand{\lN}{\lambda^\text{SU($N$)}}
%\newcommand{\Tr}{{\rm Tr\,}}
\newcommand{\bxir}{\ov{\xi}{}_R}
\newcommand{\bxil}{\ov{\xi}{}_L}
\newcommand{\xir}{\xi_R}
\newcommand{\xil}{\xi_L}
\newcommand{\bzl}{\ov{\zeta}{}_L}
\newcommand{\bzr}{\ov{\zeta}{}_R}
\newcommand{\zr}{\zeta_R}
\newcommand{\zl}{\zeta_L}
\newcommand{\nbar}{\ov{n}}

\newcommand{\CPC}{CP($N-1$)$\times$C }
\newcommand{\CPCn}{CP($N-1$)$\times$C}

\newcommand{\lar}{\lambda_R}
\newcommand{\lal}{\lambda_L}
\newcommand{\larl}{\lambda_{R,L}}
\newcommand{\lalr}{\lambda_{L,R}}
\newcommand{\bla}{\ov{\lambda}}
\newcommand{\blar}{\ov{\lambda}{}_R}
\newcommand{\blal}{\ov{\lambda}{}_L}
\newcommand{\blarl}{\ov{\lambda}{}_{R,L}}
\newcommand{\blalr}{\ov{\lambda}{}_{L,R}}

\newcommand{\bgamma}{\ov{\gamma}}
\newcommand{\bpsi}{\ov{\psi}{}}
\newcommand{\bphi}{\ov{\phi}{}}
\newcommand{\bxi}{\ov{\xi}{}}

\newcommand{\ff}{\mc{F}}
\newcommand{\bff}{\ov{\mc{F}}}

\newcommand{\eer}{\epsilon_R}
\newcommand{\eel}{\epsilon_L}
\newcommand{\eerl}{\epsilon_{R,L}}
\newcommand{\eelr}{\epsilon_{L,R}}
\newcommand{\beer}{\ov{\epsilon}{}_R}
\newcommand{\beel}{\ov{\epsilon}{}_L}
\newcommand{\beerl}{\ov{\epsilon}{}_{R,L}}
\newcommand{\beelr}{\ov{\epsilon}{}_{L,R}}

\newcommand{\bi}{{\bar \imath}}
\newcommand{\bj}{{\bar \jmath}}
\newcommand{\bk}{{\bar k}}
\newcommand{\bl}{{\bar l}}
\newcommand{\bmm}{{\bar m}}

\newcommand{\nz}{{n^{(0)}}}
\newcommand{\no}{{n^{(1)}}}
\newcommand{\bnz}{{\ov{n}{}^{(0)}}}
\newcommand{\bno}{{\ov{n}{}^{(1)}}}
\newcommand{\Dz}{{D^{(0)}}}
\newcommand{\Do}{{D^{(1)}}}
\newcommand{\bDz}{{\ov{D}{}^{(0)}}}
\newcommand{\bDo}{{\ov{D}{}^{(1)}}}
\newcommand{\sigz}{{\sigma^{(0)}}}
\newcommand{\sigo}{{\sigma^{(1)}}}
\newcommand{\bsigz}{{\ov{\sigma}{}^{(0)}}}
\newcommand{\bsigo}{{\ov{\sigma}{}^{(1)}}}

\newcommand{\rrenz}{{r_\text{ren}^{(0)}}}
\newcommand{\bren}{{\beta_\text{ren}}}

\newcommand{\mbps}{m_\text{BPS}}
\newcommand{\W}{\mathcal{W}}
\newcommand{\hsigma}{{\hat{\sigma}}}


%%%%%%%%%%%%%%%%%%%%%%%

\begin{document}

\hyphenation{con-fi-ning}
\hyphenation{Cou-lomb}
\hyphenation{Yan-ki-e-lo-wicz}

%%%%%%%%%%%%%%%%%%%%%%%%%%%%%%%%


\begin{titlepage}

\begin{flushright}
FTPI-MINN-??/??, UMN-TH-????/??\\
%January 5/2010/DRAFT
\end{flushright}

%\vspace{1cm}

\begin{center}
{  \Large \bf  Curves}
\end{center}



\vspace{2mm}


\begin{center}
{\large\bf Abstract}
\end{center}

\hspace{0.3cm}
We re-address the topic of spectrum of supersymmetric CP($N-1$) model with twisted massess...
\vspace{2cm}


\end{titlepage}


\newpage

%%%%%%%%%%%%%%%%%%%%%%%%%%%%%%%%%%%%%%%%%%%%%%%%%%%%%%%%%%%%%%%%%%%%%%%%%%%%%%%%%%
%                                                                                %
%                          I N T R O D U C T I O N                               %
%                                                                                %
%%%%%%%%%%%%%%%%%%%%%%%%%%%%%%%%%%%%%%%%%%%%%%%%%%%%%%%%%%%%%%%%%%%%%%%%%%%%%%%%%%
\section{Introduction}
\setcounter{equation}{0}


%%%%%%%%%%%%%%%%%%%%%%%%%%%%%%%%%%%%%%%%%%%%%%%%%%%%%%%%%%%%%%%%%%%%%%%%%%%%%%%%%%
%                                                                                %
%                    E X A C T  S U P E R P O T E N T I A L                      %
%                                                                                %
%%%%%%%%%%%%%%%%%%%%%%%%%%%%%%%%%%%%%%%%%%%%%%%%%%%%%%%%%%%%%%%%%%%%%%%%%%%%%%%%%%
\section{Exact Superpotential}

In the gauge formulation, the Lagrangian of \ntwoo supersymmetric CP($N-1$) model is,
\begin{align}
%
\cell & ~~=~~ 
	\frac{1}{e_0^2} \lgr \frac{1}{4} F_{\mu\nu}^2 ~+~ \left|\pt_\mu\sigma\right|^2 ~+~ \frac{1}{2}D^2
			~+~ \bla \, i\ov{\sigma}{}^\mu\pt_\mu\,\lambda \rgr
	~~+~~ i\,D\left( |n_i|^2 \;-\; 2\beta \right)
	\nonumber
	\\[3mm]
%
& 	~~+~~
	\left|\nabla_\mu n^i\right|^2 
	~~+~~ \ov{\xi}{}_i\, i\bar{\sigma}{}^\mu\nabla_\mu\,\xi^i  
	~~+~~ 2\,\sum_i \Bigl| \sigma-\frac{m_i}{\sqrt 2} \Bigr|^2\, |n^i|^2
	\nonumber
	\\[3mm]
%
&	~~+~~
	i\sqrt{2}\,\sum_i \Bigl( \sigma -\frac{m_i}{\sqrt 2}\Bigr)\, \ov{\xi}{}_{Ri}\, \xi^i_L 
	~~-~~ 
	i\sqrt{2}\,\ov{n}{}_i \left(\lambda_R\,\xi^i_L \,-\, \lambda_L\,\xi^i_R \right)
\label{fullcpn}
	\\[2mm]
%
&	~~+~~
	i\sqrt{2}\,\sum_i \Bigl( \ov{\sigma} - \frac{\ov{m}{}_i}{\sqrt 2}\Bigr)\, \ov{\xi}{}_{Li}\, \xi^i_R 
	~~-~~ 
	i\sqrt{2}\,{n}^i \left(\ov{\lambda}{}_L\,\ov{\xi}{}_{Ri} \,-\, \ov{\lambda}{}_R\,\ov{\xi}{}_{Li} \right) \,.
	\notag
\end{align}
	Here $ m_i $ are the complex twisted mass parameters. 
	The sigma model limit is obtained if $ e_0 $ is taken no infinity.

	We should mention that physically the mass parameters are not given by the masses $ m_l $ themselves,
	but rather by their differences $ m_l - m_k $ (or by $ m_l - m $, where $ m $ is the average mass).
	This is clear from Eq.~\eqref{fullcpn} as one can shift all masses by any value via a redifinition of
	$ \sigma $.

	An arbitrary choice of masses breaks the global SU($N$) invariance down to U(1)$^{N-1}$.
	However, we are interested in a special case when the masses are taken to preserve $ \mc{Z}_N $ invariance, 
	that is, when they sit on a circle,
\beq
\label{mcirc}
	m_l ~~=~~ m_0 \cdot e^{2 \pi i l / N}\,,
\eeq
	with one single complex parameter $ m_0 $.
	We will see that in this case the theory will in fact depend on $ m_0^N $.


	The theory \eqref{fullcpn} classically has $ N $ vacua, which can be seen as solutions with all $ n_i $ but one 
	equal to zero:
\begin{align}
%
	n_i & ~~=~~ (\, 0,~ ...,~ 1, ...,~ 0\, )\,,  	\qquad\qquad\qquad  k ~=~ 0,..., N-1\,.
\notag
	\\
%
	\sigma & ~~=~~ m_k \,,
\label{nvacua}
\end{align}
	Note that we chose to number both the masses and the vacua from $0$ to $N-1$, and 
	we have rescaled $ \sigma $ here,
\beq
	\sigma ~~\to~~ \frac{\sigma}{\sqrt 2}\,.
\eeq

	This theory is known to have an exact superpotential of Veneziano-Yankielowicz type 
	\cite{VYan}.
	For a theory with twisted masses the Veneziano-Yankielowicz superpotential was 
	derived in \cite{AdDVecSal,ChVa,W93,HaHo,Dor}, 
	and is obtained by integrating out the $ n^l $ fields in Eq.~\eqref{fullcpn},
\beq
\label{Wfull}
	\W_\text{eff}(\hsigma) ~~=~~
		-\, i\, \tau \hsigma ~~+~~
		\frac{1}{2\pi} \sum_j (\hsigma - m_j)\, 
				      \left\{ \ln {\frac{\hsigma - m_j}{\mu}} ~-~ 1 \right\}\,.
\eeq
	Here $ \tau $ is the complex coupling
\beq
	\tau ~~=~~ i r ~+~ \frac{\theta}{2\pi}\,, \qquad\qquad\qquad   \text{with~~~} r ~~\equiv~~ 2\beta\,,
\eeq
	and the RG scale $ \mu $ can be traded for the dynamical scale $ \Lambda $,
\beq
	\mu ~~=~~ \Lambda\, e^{2\pi r/N}\,.
\eeq
	The hat over $ \hsigma $ indicates that it actually is a (twisted) superfield.
	The vacua of this theory are found at
\beq
\label{sigvac}
	\sigma_p ~~=~~ \sqrt[N] { \Lambda^N \,+\, m_0^N } \cdot e^{ 2\pi i p / N }\,, 
\eeq
	with $ p ~=~ 0,\,...,\, N-1 $, 
	and we again assume that the masses sit on a circle.
	If $ | m_ 0 | $ is taken to be large, then it dominates over $ \Lambda $ in \eqref{sigvac},
	and the vacua take their classical values \eqref{nvacua},
\beq
	\sigma_p ~~\approx~~ m_p\,, \qquad\qquad\quad p ~=~ 0,\,...,\,N-1\,.
\eeq

	Importantly, for determination of the spectrum one needs the values of the superpotential
	in the vacuum.
	One obtains,
\beq
\label{Wvac}
	\W_\text{eff} ( \sigma_p ) ~~=~~ 
		-\, \frac{1}{2\pi}\,  
                \Bigl\{\, N\, \sigma_p ~+~ \sum_j\, m_j\, \ln \,\frac{\sigma_p - m_j}{\Lambda} \,\Bigr\}\,.
\eeq
	Then the general formula for a mass of an elementary BPS state reads as a difference of the 
	superpotential in two neighboring vacua, 
\beq
\label{mbpsgen}
	\mbps ~~=~~ \W_\text{eff} ( \sigma_{p+1} ) ~~-~~ \W_\text{eff} ( \sigma_p ) ~~+~~ i\, \vec{S} \cdot \vec{m}\,.
\eeq
	Here the difference of the vacuum superpotential values gives the topological contribution of the mass,
	while the last term, with $ \vec{S} $ an integer vector, generically gives the Noether contribution 
	due to the U(1) charge of a dyonic kink.

	In $ \mc{Z}_N $-symmetric setting of the masses \eqref{mcirc}, the theory at quantum level
	retains the $ \mc{Z}_{2N} $ symmetry, which the masses do not break.
	This symmetry manifests itself in the invariance of the spectrum 
	to the choice of the vacua in \eqref{mbpsgen}.
	We will from now on choose vacua $ \sigma_0 $ and $ \sigma_1 $ as representatives
	and focus on the masses of kinks interpolating between the two,
\beq
\label{mbpsmain}
	\mbps ~~=~~  \W_\text{eff} ( \sigma_1 ) ~~-~~ \W_\text{eff} ( \sigma_0 ) ~~+~~ i\, \vec{S} \cdot \vec{m}\,.
\eeq

	At weak coupling, the quasi-classical limit of the topological contribution is given by
	a dominating logarithm in \eqref{Wvac} in the regime $ | \Delta m | \,\gg\, \Lambda $:
\beq
\label{wkink}
	\W_\text{eff} ( \sigma_1 ) ~~-~~ \W_\text{eff} ( \sigma_0 ) ~~\sim~~
		\frac{N}{2\pi}\,
		\Delta m \cdot \ln \frac{\Delta m}{\Lambda} ~~=~~  r \cdot \Delta m\,, 
\eeq
	where $ \Delta m ~=~ m_1 \,-\, m_0 $.
 	What qualitatively differs the spectrum at weak coupling and the one at strong coupling is the allowed values of
	integer vector $ \vec{S} $.

	Classically, it is known what values vector $ \vec{S} $ takes \cite{Dor}\,,
\beq
	\vec{S} ~~=~~ (\, 0,~ ...,~ -1, ~~~...,~~ 1,~~ ...,~\, 0\, )\,,
\eeq
	with only two entries non-zero.
	Indeed, classical particle excitations have masses given by the differences $ m_k ~-~ m_l $, with the
	corresponding topological part in \eqref{mbpsmain} vanishing.
	There are also solitons and dyonic kinks, the masses of which are determined by \eqref{wkink}, 
	with $ \vec{S} $ given by their Noether U(1) charge:
\beq
\label{classdyons}
	\vec{S} ~~=~~ (\, 0,~ ...,~ -n, ~~~...,~~ n,~~ ...,~\, 0\, )\,, 
	\qquad\qquad\quad     n~\in~\mc{Z}\,.
\eeq
	This is the charge found from quantization of the global U(1) symmetry of a, strictly speaking, CP(1) soliton
	embedded into CP($N-1$) --- briefly, the symmetry rotating $ n_k $ and $ n_l $ with respect to each other.
	One therefore has an infinite tower of dyons the masses of which are given by 
	Eqs.~\eqref{mbpsmain} and \eqref{classdyons}, with an arbitrary integer $ n $.
	This is the classical lore on the weak coupling spectrum in the CP($N-1$) theory.
	What we will argue below is that this picture is incomplete. 
%	There are other $ N-2 $ U(1)'s which do not manifest themselves in such a semi-classical treatment.

\vspace{0.8cm}
	Before wrapping up this section, we make a quick mathematical comment on formula \eqref{mbpsmain}. 
	Strictly speaking, the logarithm, given by \eqref{Wvac} is a multi-branch function. 
	Therefore, the presence of the $ i\, \vec{S} \cdot \vec{m} $ term in \eqref{mbpsmain}, very
	generally, reflects the ambiguity of an expression containing a logarithm, or, 
	in other words, it fixes the branches of the logarithms. 
	Furthermore, to be as generic as possible, one would need to introduce two sets of integers 
	instead of one, corresponding to the two sets of logarithms in \eqref{mbpsmain}.
	
	This is the manifestation of the problem that the Veneziano-Yankielowicz potential, albeit being exact,
	is too ambiguous.
	One does not expect to have $ \sim N^2 $ infinite towers of BPS states 
	(there are $N$ vacua, for each of which Eq.~\eqref{Wvac} has $ N $ logarithms).
	Selection rules need to be formulated in order to restrict the set of the BPS states
	that actually exist, of the kind similar to \eqref{classdyons}. 
	We postpone the formulation of these rules until further, while for now do a simple mathematical
	trick which reduces the amount of ambiguity present in Eq.~\eqref{mbpsmain}.
	Our goal is to turn \eqref{mbpsmain} being a function of all the masses and two vacua into a function
	of a single parameter $ m_0 $.
	
	Let us pull out a factor of $ e^{2\pi i / N} $ from each term in 
	$ \W_\text{eff} ( \sigma_1 ) $ which originally looks as,
\beq
\label{Wsigone}
	\W_\text{eff} ( \sigma_1 ) ~~=~~ 
		-\, \frac{1}{2\pi}\,  
                \Bigl\{\, N\, \sigma_1 ~+~ \sum_j\, m_j\, \ln \,\frac{\sigma_1 - m_j}{\Lambda} \,\Bigr\}\,.
\eeq
	Both terms in Eq.~\eqref{Wsigone} do contain this factor.
	This move turns $ \sigma_1 $ into $ \sigma_0 $, while shifts the numeration of masses in the sum.
	To keep the numeration of masses in the sum consonant with the logarithms, we also pull out
	$ e^{2\pi i / N} $ from the argument of the logarithm.
	This constant addition vanishes when summed with $ \sum\, m_j $,
	while $ \sigma_1 $ inside the logarithm again turns into $ \sigma_0 $.
	Effectively one arrives at 
\beq
	\W_\text{eff}(\sigma_1) ~~\propto~~ e^{2\pi i / N}\, \W_\text{eff}(\sigma_0).
\eeq
	This is not the end of the story, of course, since we were not very careful with the phases
	of the logarithms, and could have easily missed (and actually did) some $ 2 \pi i $.
	This owes to the fact that $ \ln(a\,b) ~=~ \ln a ~+~ \ln b $ only modulo $ 2\pi i $.
	Let us {\it hide} these extra $ 2\pi i \cdot m_j $ coming from the logarithms into 
	the redefinition of vector $ \vec{S} $, and call the resulting vector $ \vec{Q} $.
	This vector will then strictly speaking be a different constant in different regions of the 
	complex space.
	We postpone intricacies like that until Section~\ref{sbps}, and re-write Eq.~\eqref{mbpsmain} as
\beq
\label{mspec}
	\mbps ~~=~~ \mc{U}_0 (m_0) ~~+~~ i\, \vec{Q} \cdot \vec{m}\,,
\eeq
	with an explicit function
\begin{align}
%
\label{unod}
	\mc{U}_0 (m_0) & ~~=~~ 
	\\
%
\notag
	&\!\!\!\! -\, \frac{1}{2\pi} \lgr e^{2\pi i / N} \,-\, 1 \rgr 
	\biggl\{\, N \sqrt[N] { m_0^N \,+\, \Lambda^N }  ~+~
		\sum_j\, m_j\, \ln \, \frac{ \sqrt[N] { m_0^N \,+\, \Lambda^N } \,-\, m_j } { \Lambda} \,\biggr\}\,.
\end{align}
	Here $ m_j $ is meant to be a function of $ m_0 $ as well, via \eqref{mcirc}. 
	The questions about the domain of parameter $ m_0 $ are postponed until Section~\ref{sbps} too.
	The main claim is,
\begin{itemize}
\item
	All ambiguity related to the logarithms in Eq.~\eqref{mspec} has been shifted into $ \vec{Q} $. 
	Now $ \mc{U}_0(m_0) $ is a fixed single-valued function of the complex parameter $ m_0 $.
\item
	Eq.~\eqref{mspec} has a more direct relation to the spectrum than Eq.~\eqref{mbpsmain}.
\end{itemize}
	This way we have arrived to our main formula.
	One understands that vector $ \vec{Q} $ is now different from $ \vec{S} $ in Eq.~\eqref{mbpsmain}.
	However, in the quasi-classical limit, which corresponds to large $ |m| $ {\it and} 
	large excitation number $ n $, the difference between them is negligible.
	Therefore, in that limit, vector $ \vec{Q} $ is expected to take the form akin to the 
	RHS of Eq.~\eqref{classdyons},
\beq
\label{qclass}
	\vec{Q} ~~=~~ (\, 0,~ ...,~ -n, ~~~...,~~ n,~~ ...,~\, 0\, )\,, 
	\qquad\qquad\quad     n~\in~\mc{Z}\,.
\eeq



%%%%%%%%%%%%%%%%%%%%%%%%%%%%%%%%%%%%%%%%%%%%%%%%%%%%%%%%%%%%%%%%%%%%%%%%%%%%%%%%%%
%                                                                                %
%                         M I R R O R  T R E A T M E N T                         %
%                                                                                %
%%%%%%%%%%%%%%%%%%%%%%%%%%%%%%%%%%%%%%%%%%%%%%%%%%%%%%%%%%%%%%%%%%%%%%%%%%%%%%%%%%
\section{Mirror Treatment}

	Now we gradually pass to the discussion of what is known 
	about the strong coupling spectrum.
	There are $ N $ BPS kinks.
	They can be seen in the mirror representation \cite{MR1}, 
\beq
\label{mirror}
	\W_\text{mirror}^\text{CP($N-1$)} ~~=~~
		-\, \frac{\Lambda}{4\pi}\, 
		\Bigl\{\, \sum_j X_j ~~-~~ \sum_j \frac{m_j}{\Lambda}\, \ln X_j \,\Bigr\}\,,
\eeq
	with
\beq
	\prod_j\, X_j ~~=~~ 1\,.
\eeq
	As shown in \cite{Shifman:2010id}, one can determine the masses of the $ N $ kinks
	near the origin, $ | m_j | ~\ll~ \Lambda $:
\beq
\label{mirrorm}
	\mbps ~~\approx~~ \frac{N}{2\pi} \lgr e^{2\pi i / N} \,-\, 1 \rgr \Lambda
			   ~~+~~ i\, ( m_j \,-\, m )\,,
\eeq
	where $ m $ is the average mass, vanishing in the $ \mc{Z}_N $ case.
	So, to the linear order in the mass parameter, one has $ N $ kinks with the masses given
	by a large $ \Lambda $ term, and the splittings determined by $ m_j $ themselves,
\beq
\label{smirror}
	\mbps ~~\approx~~\frac{N}{2\pi} \lgr e^{2\pi i / N} \,-\, 1 \rgr \Lambda
			   ~~+~~ i\, m_j\,,
	\qquad\quad j~=~ 0,\,...,\, N-1\,.
\eeq

%%%%%%%%%%%%%%%%%%%%%%%%%%%%%%%%%%%%%%%%%%%%%%%%%%%%%%%%%%%%%%%%%%%%%%%%%%%%%%%%%%
%                                                                                %
%                     A R G Y R E S - D O U G L A S  P O I N T                   %
%                                                                                %
%%%%%%%%%%%%%%%%%%%%%%%%%%%%%%%%%%%%%%%%%%%%%%%%%%%%%%%%%%%%%%%%%%%%%%%%%%%%%%%%%%
\section{Argyres-Douglas Point}

	If all masses $ m_j $ sit on the circle, we have argued that the corresponding
	vacua will also be forced to sit on the circle, as governed by the exact superpotential
	\eqref{Wfull}.
	There are therefore only simultaneous collisions of all vacua $ \sigma_p $
	in the theory.
	The Argyres-Douglas points \cite{AD} will therefore correspond to 
\beq
	\sigma_p ~~=~~ 0\,.
\eeq
	This occurs whenever $ m_0 $ equals $ \Lambda $ times an $N$-th root of $ -1 $,
\beq
	m_0^\text{AD} ~~=~~ \Lambda \, e^{i \pi / N} \cdot e^{2\pi i l / N}\,,
	\qquad\qquad\quad 
	l ~=~ 0,\,...,\, N-1 \,.
\eeq
	In particular, the most convenient for us will be the two AD points closest to the
	real positive axis:
\beq
	m_0^\text{AD} ~~=~~ \Lambda \, e^{i \pi / N}
	\qquad\quad
	\text{and}
	\qquad\quad
	m_0^\text{AD} ~~=~~ \Lambda \, e^{- i \pi / N}\,.
\eeq

	One might wonder why we make a distinction between the different AD points, when
	in the physical complex $ m^N $ plane they all blend into one same point $ m^N ~=~ - 1 $
	(in units of $ \Lambda $).
	The reason is that while physically there is indeed only one AD, in $ m^0 $ plane
	one actually gains a monodromy when passing from one AD point to another \cite{Dor,ls1}.
	We would like to observe that monodromy.
	When needed, one can always resort to the physical $ m^N $ plane by raising the 
	results from $ m^0 $ plane to the $ N $-th power (but not the other way round).

	The crucial observation about the AD point is that one of the soliton states becomes massless
	at that location.
	Briefly, if all the vacua merge at the AD point, then Eq.~\eqref{mbpsmain} tells one that
	the kink with $ \vec{S} ~=~ 0 $ becomes massless 
\beq
	\mbps ~~=~~ \W_\text{eff}(\sigma_1) ~-~ \W_\text{eff}(\sigma_0) ~~=~~ 0\,.
\eeq
	We quote this as a qualitative statement and render it more precise in Section~\ref{sbps}.

%%%%%%%%%%%%%%%%%%%%%%%%%%%%%%%%%%%%%%%%%%%%%%%%%%%%%%%%%%%%%%%%%%%%%%%%%%%%%%%%%%
%                                                                                %
%                           B P S  S P E C T R U M                               %
%                                                                                %
%%%%%%%%%%%%%%%%%%%%%%%%%%%%%%%%%%%%%%%%%%%%%%%%%%%%%%%%%%%%%%%%%%%%%%%%%%%%%%%%%%
\section{BPS Spectrum}
\label{sbps}

	We now turn to the discussion of the BPS spectrum in the strong coupling region.
	Surprisingly, the conclusions obtained in the strong coupling sector will allow us
	to make implications for the weak coupling sector as well.
	We first collect the results known and trustworthy about CP(1) theory,
	as the simplest and well-studied case, and then increase $ N $.

%%%%%%%%%%%%%%%%%%%%%%%%%%%%%%%%%%%%%%%%%%%%%%%%%%%%%%%%%%%%%%%%%%%%%%%%%%%%%%%%%%
%%%%%%%%%%%%%%%%%%%%%%%%%%%%%%%%%%%%%%%%%%%%%%%%%%%%%%%%%%%%%%%%%%%%%%%%%%%%%%%%%%
\subsection{BPS Spectrum in CP(1) at Strong Coupling}

	There are two kinks in the strong coupling sector with quantum numbers
	$ (T, Q) $ $ = (1, 0) $ and $ (T, Q) $ $ = (1, 1) $.
	Here $ T $ bears the convential meaning of the topological charge, the
	constant that multiplies $ \W_\text{eff}(\sigma_1) \,-\, \W_\text{eff}(\sigma_0) $,
	and $ Q \in \mc{Z} $ is the U(1) charge, {\it i.e.} the one from Eq.~\eqref{qclass}.

	The formula for the BPS mass in CP(1) theory is well-known. 
	We re-derive it from our master equations \eqref{mspec} and \eqref{unod}.
	Instead of $ m_0 $, it conventionally is written in terms of the mass difference $ \Delta m $ 
	$ ~=~ m_1 - m_0 $ $~=~ - 2 m_0 $,	
\beq
\label{mcp1}
	\mbps^\text{CP(1)} ~~=~~
	\frac{1}{\pi} \lgr   \sqrt{ \Delta m^2 \,+\, 4 \Lambda^2\, }
			~-~ \frac{\Delta m}{2}\, 
			    \ln\, \frac {\Delta m \,+\, \sqrt{ \Delta m^2 \,+\, 4 \Lambda^2\, }}
                                        {\Delta m \,-\, \sqrt{ \Delta m^2 \,+\, 4 \Lambda^2\, }}
                      \rgr
	\!~~+~~
	i\,\vec{Q} \cdot \vec{m}\,.
\eeq
	The Argyres-Douglas points here are $ \Delta m ~=~ \pm\, 2i \Lambda $, and correspond to
	the vanishing square root.
	Let us exhibit the monodromy which occurs when one moves from one AD point to the other.
	Say, choose the branch of the logarithm such that at the point $ \Delta m ~=~ 2i \Lambda $ 
	the logarithm vanishes.
	Then the kink $ \vec{Q} ~=~ 0 $ becomes massless at that point. 
	Let us move from the point $ 2i \Lambda $ to $ - 2i \Lambda $ along a large circle, see
	Fig~\ref{contour_Dm}a.
\begin{figure}
\begin{center}
\begin{tabular}{cc}
%
\epsfxsize=6cm
 \epsfbox{contour_Dm.epsi}
~~
 &
\epsfxsize=5.2cm
 \epsfbox{contour_ln.epsi}
~~
  \\
%
	(a)  &  (b)
\end{tabular}
\end{center}
\caption{(a) The large-radius contour in the $ \Delta m $ plane, starting at the AD point $ 2i \Lambda $
	and terminating at the point $ -2i \Lambda $.
	 (b) The same contour shown in the plane of 
	$ z ~=~ \frac {\Delta m \,+\, \sqrt{ \Delta m^2 \,+\, 4 \Lambda^2\, }}
                      {\Delta m \,-\, \sqrt{ \Delta m^2 \,+\, 4 \Lambda^2\, }} $
	--- the argument of the logarithm in Eq.~\eqref{mcp1}.}
\label{contour_Dm}
\end{figure}
	It is easy to show that the argument of the logarithm in Eq.~\eqref{mcp1} will also sketch
	a large circle, starting and terminating at one, see Fig~\ref{contour_Dm}b.
	However, this contour arrives at the previous branch of the logarithm compared to where 
	it started from.
	The argument of the logarithm is forced to experience a discontinuity of $ + 2 \pi i $
	needed for $ \Delta m $ to hold on the physical branch. 
	As a result, the masses of kinks now shift compared to $ \Delta m ~=~ +\, 2 i \Lambda $ point,
\beq
	m_\text{BPS}^{CP(1)}( -\, 2 i \Lambda ) ~~=~~ - \, i\, \Delta m ~~+~~ i\,\vec{Q} \cdot \vec{m}\,.
\eeq
	One observes, that it is the $ (T, Q) $ $ = (1, 1) $ kink which becomes massless 
	(in terms of $ \vec{Q} $, it is the $ \vec{Q} ~=~ (-1,~ 1) $ kink).

	How does it become that at weak coupling one has the whole tower of states
	while at the strong coupling there are only two?
	Similar to what it occurs in the Seiberg-Witten theory, the states of the weak coupling sector
	decay on the curves of marginal stability \cite{Bilal:1996sk,Bilal:1997st}.
	Only two states survive when crossing into the strong coupling region, 
	and those are precisely the states which become massless at the Argyres-Douglas points.

	Finally for CP(1) let us show that Eq.~\eqref{mcp1} is coherent with the result from the 
	mirror theory \eqref{smirror}.
	At small $ \Delta m $, the argument of the logarithm in Eq.~\eqref{mcp1} just turns into $ -1 $, 
	while the first term in the bracket can be rounded to $ \Lambda $.
	Re-writing $ i\, \vec{Q}\cdot\vec{m} $ as $ i\, Q \cdot \Delta m $ with $ Q = 0 \text{~or~} 1 $, 
	we come to
\beq
	m_\text{BPS}^{CP(1)} ~~\approx~~ 
		\frac{1}{\pi}\, 2\Lambda  ~~+~~ i\, \Bigl(Q - \frac{1}{2}\Bigr)\, \Delta m\,.
\eeq
	It is easy to see that this is indeed the spectrum predicted by Eq.~\eqref{smirror} to the 
	linear order in $ \Delta m $.

\vspace{0.8cm}
	We can now formulate the {\it selection rules} for the spectrum of BPS states in the overall
	region of the complex mass parameter $ m_0 $:
\begin{itemize}
\item
	{\it Quasi-classical limit} --- the spectrum at large $ m_0 $ and large excitation number $ n $
	must reproduce the semi-classical result \eqref{wkink}, \eqref{qclass}:
\beq
\mbps ~~\simeq~~ \frac{N}{2\pi}\,
		\Delta m \cdot \ln \frac{\Delta m}{\Lambda}  
	    ~~+~~
	i\, Q \cdot ( m_1 \,-\, m_0 )\,
\eeq

\item
	{\it Argyres-Douglas point} --- the only states that survive when crossing from weak coupling 
	into the strong coupling region are those $ N $ states which become massless at the AD points

\item
	{\it Mirror spectrum} --- the latter $ N $ kinks must form the spectrum given by mirror 
	formula \eqref{smirror} in the small $ m_0 $ limit 
\end{itemize}

\vspace{2.0cm}

%%%%%%%%%%%%%%%%%%%%%%%%%%%%%%%%%%%%%%%%%%%%%%%%%%%%%%%%%%%%%%%%%%%%%%%%%%%%%%%%%%
%%%%%%%%%%%%%%%%%%%%%%%%%%%%%%%%%%%%%%%%%%%%%%%%%%%%%%%%%%%%%%%%%%%%%%%%%%%%%%%%%%
\subsection{Strong Coupling Spectrum in CP(2)}
\vspace{2.0cm}

%%%%%%%%%%%%%%%%%%%%%%%%%%%%%%%%%%%%%%%%%%%%%%%%%%%%%%%%%%%%%%%%%%%%%%%%%%%%%%%%%%
%%%%%%%%%%%%%%%%%%%%%%%%%%%%%%%%%%%%%%%%%%%%%%%%%%%%%%%%%%%%%%%%%%%%%%%%%%%%%%%%%%
\subsection{Spectrum in CP($N-1$)}
\vspace{2.0cm}


%%%%%%%%%%%%%%%%%%%%%%%%%%%%%%%%%%%%%%%%%%%%%%%%%%%%%%%%%%%%%%%%%%%%%%%%%%%%%%%%%%
%                                                                                %
%            C U R V E S  O F  M A R G I N A L  S T A B I L I T Y                %
%                                                                                %
%%%%%%%%%%%%%%%%%%%%%%%%%%%%%%%%%%%%%%%%%%%%%%%%%%%%%%%%%%%%%%%%%%%%%%%%%%%%%%%%%%
\section{Curves of Marginal Stability}

\begin{itemize}

\item
	$ N - 1 $ towers ~~$ \Longrightarrow $~~ $ N - 1 $ curves

\item
	1st curve passes through the AD point;\\
	expansion near the AD point

\item
	large-$N$ limit, from expansion in the right half-plane

\end{itemize}
\vspace{2.0cm}

%%%%%%%%%%%%%%%%%%%%%%%%%%%%%%%%%%%%%%%%%%%%%%%%%%%%%%%%%%%%%%%%%%%%%%%%%%%%%%%%%%
%                                                                                %
%                            C O N C L U S I O N                                 %
%                                                                                %
%%%%%%%%%%%%%%%%%%%%%%%%%%%%%%%%%%%%%%%%%%%%%%%%%%%%%%%%%%%%%%%%%%%%%%%%%%%%%%%%%%
\section{Conclusion}

\begin{itemize}
\item
	$ N - 1 $ towers which quasiclassically all blend and cannot be resolved;\\
	only one tower is seen classically

\item
	Consequences for SQCD

\end{itemize}
\vspace{2.0cm}

\begin{thebibliography}{99}

\bibitem{Dor}
N.~Dorey,
%``The BPS spectra of two-dimensional
%supersymmetric gauge theories
%with  twisted mass terms,''
JHEP {\bf 9811}, 005 (1998) [hep-th/9806056].
%%CITATION = HEP-TH 9806056;%%

\bibitem{MR1}
  K.~Hori and C.~Vafa,
{\em Mirror symmetry,}
  arXiv:hep-th/0002222.
  %%CITATION = HEP-TH/0002222;%%
  
\bibitem{MR2}
E.~Frenkel and A.~Losev,
  %``Mirror symmetry in two steps: A-I-B,''
  Commun.\ Math.\ Phys.\  {\bf 269}, 39 (2006)
  [arXiv:hep-th/0505131].
  %%CITATION = CMPHA,269,39;%%

%\cite{Shifman:2010id}
\bibitem{Shifman:2010id}
  M.~Shifman, A.~Yung,
  %``Non-Abelian Confinement in N=2 Supersymmetric QCD: Duality and Kinks on Confining Strings,''
  Phys.\ Rev.\  {\bf D81}, 085009 (2010).
  [arXiv:1002.0322 [hep-th]].

\bibitem{ls1}
  M.~Shifman, A.~Vainshtein and R.~Zwicky,
  %``Central charge anomalies in 2D sigma models with twisted mass,''
  J.\ Phys.\ A  {\bf 39}, 13005 (2006)
  [arXiv:hep-th/0602004].
  %%CITATION = JPAGB,A39,13005;%

%\cite{Olmez:2007sg}
\bibitem{Olmez}
  S.~\"{O}lmez and M.~Shifman,
  %``Curves of Marginal Stability in Two-Dimensional CP(N-1) Models with
  %Z_N-Symmetric Twisted Masses,''
  J.\ Phys.\ A  {\bf 40}, 11151 (2007)
  [arXiv:hep-th/0703149].
  %%CITATION = JPAGB,A40,11151;%%

\bibitem{VYan}
 G.~Veneziano and S.~Yankielowicz,
  %``An Effective Lagrangian For The Pure N=1 Supersymmetric Yang-Mills
  %Theory,''
  Phys.\ Lett.\  B {\bf 113}, 231 (1982).
  %%CITATION = PHLTA,B113,231;%%

\bibitem{AdDVecSal}
A.~D'Adda, A.~C.~Davis, P.~DiVeccia and P.~Salamonson,
%"An effective action for the supersymmetric CP$^{n-1}$ models,"
Nucl.\ Phys.\ {\bf B222} 45 (1983).

\bibitem{ChVa}
S.~Cecotti and C. Vafa,
%"On classification of \ntwo supersymmetric theories,"
Comm. \ Math. \ Phys. \ {\bf 158} 569 (1993)
[hep-th/9211097].

\bibitem{W93}
E.~Witten,
  %``Phases of N = 2 theories in two dimensions,''
  Nucl.\ Phys.\ B {\bf 403}, 159 (1993)
  [hep-th/9301042].
  %%CITATION = HEP-TH 9301042;%%

\bibitem{HaHo}
A.~Hanany and K.~Hori,
  %``Branes and N = 2 theories in two dimensions,''
  Nucl.\ Phys.\  B {\bf 513}, 119 (1998)
  [arXiv:hep-th/9707192].
  %%CITATION = NUPHA,B513,119;%%

\bibitem{AD}
P. C.~Argyres and M. R.~Douglas,
%``New Phenomena in SU(3) Supersymmetric Gauge Theory'' 
Nucl. \ Phys. \ {\bf B448}, 93 (1995)   
[arXiv:hep-th/9505062].
%%CITATION = NUPHA,B448,93;%%
  
%\cite{Bilal:1996sk}
\bibitem{Bilal:1996sk}
  A.~Bilal, F.~Ferrari,
  %``Curves of marginal stability, and weak and strong coupling BPS spectra in N=2 supersymmetric QCD,''
  Nucl.\ Phys.\  {\bf B480}, 589-622 (1996).
  [hep-th/9605101].

%\cite{Bilal:1997st}
\bibitem{Bilal:1997st}
  A.~Bilal, F.~Ferrari,
  %``The BPS spectra and superconformal points in massive N=2 supersymmetric QCD,''
  Nucl.\ Phys.\  {\bf B516}, 175-228 (1998).
  [hep-th/9706145].

\end{thebibliography}


\end{document}
