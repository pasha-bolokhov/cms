%\documentclass{article}
\documentclass[12pt]{article}
\usepackage{latexsym}
\usepackage{amsmath}
\usepackage{amssymb}
\usepackage{relsize}
\usepackage{geometry}
\geometry{letterpaper}

\usepackage{showlabels}

\textwidth = 6.0 in
\textheight = 8.5 in
\oddsidemargin = 0.0 in
\evensidemargin = 0.0 in
\topmargin = 0.2 in
\headheight = 0.0 in
\headsep = 0.0 in
%\parskip = 0.05in
\parindent = 0.35in


%% common definitions
\def\stackunder#1#2{\mathrel{\mathop{#2}\limits_{#1}}}
\def\beqn{\begin{eqnarray}}
\def\eeqn{\end{eqnarray}}
\def\nn{\nonumber}
\def\baselinestretch{1.1}
\def\beq{\begin{equation}}
\def\eeq{\end{equation}}
\def\ba{\beq\new\begin{array}{c}}
\def\ea{\end{array}\eeq}
\def\be{\ba}
\def\ee{\ea}
\def\stackreb#1#2{\mathrel{\mathop{#2}\limits_{#1}}}
\def\Tr{{\rm Tr}}
\newcommand{\gsim}{\lower.7ex\hbox{$
\;\stackrel{\textstyle>}{\sim}\;$}}
\newcommand{\lsim}{\lower.7ex\hbox{$
\;\stackrel{\textstyle<}{\sim}\;$}}
%%%%%%%%%%
\newcommand{\nfour}{${\mathcal N}=4$ }
\newcommand{\ntwo}{${\mathcal N}=2$ }
\newcommand{\ntwon}{${\mathcal N}=2$}
\newcommand{\ntwot}{${\mathcal N}= \left(2,2\right) $ }
\newcommand{\ntwoo}{${\mathcal N}= \left(0,2\right) $ }
\newcommand{\ntwoon}{${\mathcal N}= \left(0,2\right) $}
\newcommand{\none}{${\mathcal N}=1$ }
%%%%%%%%%%%
\newcommand{\nonen}{${\mathcal N}=1$}
\newcommand{\vp}{\varphi}
\newcommand{\pt}{\partial}
\newcommand{\ve}{\varepsilon}
\newcommand{\gs}{g^{2}}
\newcommand{\qt}{\tilde q}
%\renewcommand{\theequation}{\thesection.\arabic{equation}}

%%
\newcommand{\p}{\partial}
\newcommand{\wt}{\widetilde}
\newcommand{\ov}{\overline}
\newcommand{\mc}[1]{\mathcal{#1}}
\newcommand{\md}{\mathcal{D}}

\newcommand{\GeV}{{\rm GeV}}
\newcommand{\eV}{{\rm eV}}
\newcommand{\Heff}{{\mathcal{H}_{\rm eff}}}
\newcommand{\Leff}{{\mathcal{L}_{\rm eff}}}
\newcommand{\el}{{\rm EM}}
\newcommand{\uflavor}{\mathbf{1}_{\rm flavor}}
\newcommand{\lgr}{\left\lgroup}
\newcommand{\rgr}{\right\rgroup}

\newcommand{\Mpl}{M_{\rm Pl}}
\newcommand{\suc}{{{\rm SU}_{\rm C}(3)}}
\newcommand{\sul}{{{\rm SU}_{\rm L}(2)}}
\newcommand{\sutw}{{\rm SU}(2)}
\newcommand{\suth}{{\rm SU}(3)}
\newcommand{\ue}{{\rm U}(1)}
%%%%%%%%%%%%%%%%%%%%%%%%%%%%%%%%%%%%%%%
%  Slash character...
\def\slashed#1{\setbox0=\hbox{$#1$}             % set a box for #1
   \dimen0=\wd0                                 % and get its size
   \setbox1=\hbox{/} \dimen1=\wd1               % get size of /
   \ifdim\dimen0>\dimen1                        % #1 is bigger
      \rlap{\hbox to \dimen0{\hfil/\hfil}}      % so center / in box
      #1                                        % and print #1
   \else                                        % / is bigger
      \rlap{\hbox to \dimen1{\hfil$#1$\hfil}}   % so center #1
      /                                         % and print /
   \fi}                                        %

%%EXAMPLE:  $\slashed{E}$ or $\slashed{E}_{t}$

%%

\newcommand{\LN}{\Lambda_\text{SU($N$)}}
\newcommand{\sunu}{{\rm SU($N$) $\times$ U(1) }}
\newcommand{\sunun}{{\rm SU($N$) $\times$ U(1)}}
\def\cfl {$\text{SU($N$)}_{\rm C+F}$ }
\def\cfln {$\text{SU($N$)}_{\rm C+F}$}
\newcommand{\mUp}{m_{\rm U(1)}^{+}}
\newcommand{\mUm}{m_{\rm U(1)}^{-}}
\newcommand{\mNp}{m_\text{SU($N$)}^{+}}
\newcommand{\mNm}{m_\text{SU($N$)}^{-}}
\newcommand{\AU}{\mc{A}^{\rm U(1)}}
\newcommand{\AN}{\mc{A}^\text{SU($N$)}}
\newcommand{\aU}{a^{\rm U(1)}}
\newcommand{\aN}{a^\text{SU($N$)}}
\newcommand{\baU}{\ov{a}{}^{\rm U(1)}}
\newcommand{\baN}{\ov{a}{}^\text{SU($N$)}}
\newcommand{\lU}{\lambda^{\rm U(1)}}
\newcommand{\lN}{\lambda^\text{SU($N$)}}
%\newcommand{\Tr}{{\rm Tr\,}}
\newcommand{\bxir}{\ov{\xi}{}_R}
\newcommand{\bxil}{\ov{\xi}{}_L}
\newcommand{\xir}{\xi_R}
\newcommand{\xil}{\xi_L}
\newcommand{\bzl}{\ov{\zeta}{}_L}
\newcommand{\bzr}{\ov{\zeta}{}_R}
\newcommand{\zr}{\zeta_R}
\newcommand{\zl}{\zeta_L}
\newcommand{\nbar}{\ov{n}}

\newcommand{\ssm}{{\scriptscriptstyle(M)}}
\newcommand{\sse}{{\scriptscriptstyle(E)}}
\newcommand{\cell}{{\mathcal L}}
\newcommand{\CPC}{CP($N-1$)$\times$C }
\newcommand{\CPCn}{CP($N-1$)$\times$C}
\newcommand{\cpn}{CP$(N-1)\,$}

\newcommand{\lar}{\lambda_R}
\newcommand{\lal}{\lambda_L}
\newcommand{\larl}{\lambda_{R,L}}
\newcommand{\lalr}{\lambda_{L,R}}
\newcommand{\blar}{\ov{\lambda}{}_R}
\newcommand{\blal}{\ov{\lambda}{}_L}
\newcommand{\blarl}{\ov{\lambda}{}_{R,L}}
\newcommand{\blalr}{\ov{\lambda}{}_{L,R}}

\newcommand{\tgamma}{\wt{\gamma}}
\newcommand{\btgamma}{\ov{\tgamma}}
\newcommand{\bpsi}{\ov{\psi}{}}
\newcommand{\bphi}{\ov{\phi}{}}
\newcommand{\bxi}{\ov{\xi}{}}

\newcommand{\ff}{\mc{F}}
\newcommand{\bff}{\ov{\mc{F}}}

\newcommand{\eer}{\epsilon_R}
\newcommand{\eel}{\epsilon_L}
\newcommand{\eerl}{\epsilon_{R,L}}
\newcommand{\eelr}{\epsilon_{L,R}}
\newcommand{\beer}{\ov{\epsilon}{}_R}
\newcommand{\beel}{\ov{\epsilon}{}_L}
\newcommand{\beerl}{\ov{\epsilon}{}_{R,L}}
\newcommand{\beelr}{\ov{\epsilon}{}_{L,R}}

\newcommand{\bi}{{\bar \imath}}
\newcommand{\bj}{{\bar \jmath}}
\newcommand{\bk}{{\bar k}}
\newcommand{\bl}{{\bar l}}
\newcommand{\bm}{{\bar m}}

\begin{document}


\begin{center}
\Large\bf
	Response to the Referee's Comment to DE10856 
\end{center}

	We agree with the referee: the towers of states in the spectrum that had been derived in our paper 
	need a  semiclassical check. In this particular investigation we focused on a combination of
	general features of the ${\mathcal N}=2$ CP($N-1$) model and strong coupling results due to 
	Vafa and collaborators to predict this spectrum per se for the $Z_N$ symmetric mass choice.
	This had never been done before. To our mind, the result is important enough to warrant its publication.
	
	A complete semiclassical analysis is a full-fledged task, 
	which we plan to address in a separate project which will have a wider scope than the one mentioned by the referee. 
	Besides the obvious task of finding and properly interpreting the quantum corrections to the masses of the solitons
	in CP($N-1$), 
	we will also study, in parallel, the ``progenitor" states in the four-dimensional
	gauge theory, where --- we believe --- the same spectrum holds at the root of the baryonic Higgs branch.
	We agree that the new states we have discovered closely hint to the bound states of the monopoles and quarks. 
	Such states in the CP($N-1$) theory were discussed in \cite{Dorey:1999zk}.
	Whether our states are indeed of this type  is an important question and part of this problem as well.
	Regardless of this, implications for four dimensions inferred from the two-dimensional mass spectrum 
	must have fundamental significance for the gauge theory. Thus, this is going to be a large project by itself, 
	and we see no reason 
	to squeeze it in the present paper.

	However, to make a step forward to the referee and to facilitate the reader's task
	we  add  a brief overview section ``Prospects for the Spectrum in QuasiClassics''. 
	The newly found towers of states differ from each other (as well as from the previously known one \cite{Dor})
	by fractional U(1) charge contributions, which occur due to the presence of fermions. 
	We note that such a contribution is not new to this theory in the quasiclassical regime, and is known
	 for instance in CP(1) \cite{ls1,Shifman:2005rs}.
	Although in the latter theory this does not lead to the emergence of new states, it 
	exhibits the effect in a transparent manner and
	renders it very plausible that 
	the same effect is responsible for the mass splitting in CP($N-1$).
	
	Overall, we believe that the semiclassical derivation of the new spectrum is a multi-faceted problem, 
	and an appropriate topic for discussion in a separate work.
	
\begin{thebibliography}{99}
%\cite{Dorey:1999zk}
\bibitem{Dorey:1999zk}
  N.~Dorey, T.~J.~Hollowood, D.~Tong,
  %``The BPS spectra of gauge theories in two-dimensions and four-dimensions,''
  JHEP {\bf 9905}, 006 (1999).
  [hep-th/9902134].

\bibitem{Dor}
N.~Dorey,
%``The BPS spectra of two-dimensional
%supersymmetric gauge theories
%with  twisted mass terms,''
JHEP {\bf 9811}, 005 (1998) [hep-th/9806056].
%%CITATION = HEP-TH 9806056;%%
  
\bibitem{ls1}
  M.~Shifman, A.~Vainshtein, R.~Zwicky,
  %``Central charge anomalies in 2-D sigma models with twisted mass,''
  J.\ Phys.\ A {\bf A39}, 13005-13024 (2006).
  [hep-th/0602004].
  
%\cite{Shifman:2005rs}
\bibitem{Shifman:2005rs}
  M.~Shifman,
  %``Supersymmetric solitons and topology,''
  Lect.\ Notes Phys.\  {\bf 659}, 237-284 (2005).

\end{thebibliography}


\end{document}

