%\documentclass{article}
\documentclass[12pt]{article}
\usepackage{latexsym}
\usepackage{amsmath}
\usepackage{amssymb}
\usepackage{relsize}
\usepackage{geometry}
\geometry{letterpaper}

%\usepackage{showlabels}

\textwidth = 6.0 in
\textheight = 8.5 in
\oddsidemargin = 0.0 in
\evensidemargin = 0.0 in
\topmargin = 0.2 in
\headheight = 0.0 in
\headsep = 0.0 in
%\parskip = 0.05in
\parindent = 0.35in


%% common definitions
\def\stackunder#1#2{\mathrel{\mathop{#2}\limits_{#1}}}
\def\beqn{\begin{eqnarray}}
\def\eeqn{\end{eqnarray}}
\def\nn{\nonumber}
\def\baselinestretch{1.1}
\def\beq{\begin{equation}}
\def\eeq{\end{equation}}
\def\ba{\beq\new\begin{array}{c}}
\def\ea{\end{array}\eeq}
\def\be{\ba}
\def\ee{\ea}
\def\stackreb#1#2{\mathrel{\mathop{#2}\limits_{#1}}}
\def\Tr{{\rm Tr}}
\newcommand{\gsim}{\lower.7ex\hbox{$
\;\stackrel{\textstyle>}{\sim}\;$}}
\newcommand{\lsim}{\lower.7ex\hbox{$
\;\stackrel{\textstyle<}{\sim}\;$}}
%%%%%%%%%%
\newcommand{\nfour}{${\mathcal N}=4$ }
\newcommand{\ntwo}{${\mathcal N}=2$ }
\newcommand{\ntwon}{${\mathcal N}=2$}
\newcommand{\ntwot}{${\mathcal N}= \left(2,2\right) $ }
\newcommand{\ntwoo}{${\mathcal N}= \left(0,2\right) $ }
\newcommand{\ntwoon}{${\mathcal N}= \left(0,2\right) $}
\newcommand{\none}{${\mathcal N}=1$ }
%%%%%%%%%%%
\newcommand{\nonen}{${\mathcal N}=1$}
\newcommand{\vp}{\varphi}
\newcommand{\pt}{\partial}
\newcommand{\ve}{\varepsilon}
\newcommand{\gs}{g^{2}}
\newcommand{\qt}{\tilde q}
%\renewcommand{\theequation}{\thesection.\arabic{equation}}

%%
\newcommand{\p}{\partial}
\newcommand{\wt}{\widetilde}
\newcommand{\ov}{\overline}
\newcommand{\mc}[1]{\mathcal{#1}}
\newcommand{\md}{\mathcal{D}}

\newcommand{\GeV}{{\rm GeV}}
\newcommand{\eV}{{\rm eV}}
\newcommand{\Heff}{{\mathcal{H}_{\rm eff}}}
\newcommand{\Leff}{{\mathcal{L}_{\rm eff}}}
\newcommand{\el}{{\rm EM}}
\newcommand{\uflavor}{\mathbf{1}_{\rm flavor}}
\newcommand{\lgr}{\left\lgroup}
\newcommand{\rgr}{\right\rgroup}

\newcommand{\Mpl}{M_{\rm Pl}}
\newcommand{\suc}{{{\rm SU}_{\rm C}(3)}}
\newcommand{\sul}{{{\rm SU}_{\rm L}(2)}}
\newcommand{\sutw}{{\rm SU}(2)}
\newcommand{\suth}{{\rm SU}(3)}
\newcommand{\ue}{{\rm U}(1)}

%%%%%%%%%%%%%%%%%%%%%%%%%%%%%%%%%%%%%%%
%  Slash character...
\def\slashed#1{\setbox0=\hbox{$#1$}             % set a box for #1
   \dimen0=\wd0                                 % and get its size
   \setbox1=\hbox{/} \dimen1=\wd1               % get size of /
   \ifdim\dimen0>\dimen1                        % #1 is bigger
      \rlap{\hbox to \dimen0{\hfil/\hfil}}      % so center / in box
      #1                                        % and print #1
   \else                                        % / is bigger
      \rlap{\hbox to \dimen1{\hfil$#1$\hfil}}   % so center #1
      /                                         % and print /
   \fi}                                         %
%%EXAMPLE:  $\slashed{E}$ or $\slashed{E}_{t}$

%%

\newcommand{\LN}{\Lambda_\text{SU($N$)}}
\newcommand{\sunu}{{\rm SU($N$) $\times$ U(1) }}
\newcommand{\sunun}{{\rm SU($N$) $\times$ U(1)}}
\def\cfl {$\text{SU($N$)}_{\rm C+F}$ }
\def\cfln {$\text{SU($N$)}_{\rm C+F}$}
\newcommand{\mUp}{m_{\rm U(1)}^{+}}
\newcommand{\mUm}{m_{\rm U(1)}^{-}}
\newcommand{\mNp}{m_\text{SU($N$)}^{+}}
\newcommand{\mNm}{m_\text{SU($N$)}^{-}}
\newcommand{\AU}{\mc{A}^{\rm U(1)}}
\newcommand{\AN}{\mc{A}^\text{SU($N$)}}
\newcommand{\aU}{a^{\rm U(1)}}
\newcommand{\aN}{a^\text{SU($N$)}}
\newcommand{\baU}{\ov{a}{}^{\rm U(1)}}
\newcommand{\baN}{\ov{a}{}^\text{SU($N$)}}
\newcommand{\lU}{\lambda^{\rm U(1)}}
\newcommand{\lN}{\lambda^\text{SU($N$)}}
%\newcommand{\Tr}{{\rm Tr\,}}
\newcommand{\bxir}{\ov{\xi}{}_R}
\newcommand{\bxil}{\ov{\xi}{}_L}
\newcommand{\xir}{\xi_R}
\newcommand{\xil}{\xi_L}
\newcommand{\bzl}{\ov{\zeta}{}_L}
\newcommand{\bzr}{\ov{\zeta}{}_R}
\newcommand{\zr}{\zeta_R}
\newcommand{\zl}{\zeta_L}
\newcommand{\nbar}{\ov{n}}

\newcommand{\ssm}{{\scriptscriptstyle(M)}}
\newcommand{\sse}{{\scriptscriptstyle(E)}}
\newcommand{\cell}{{\mathcal L}}
\newcommand{\CPC}{CP($N-1$)$\times$C }
\newcommand{\CPCn}{CP($N-1$)$\times$C}
\newcommand{\cpn}{CP$(N-1)\,$}

\newcommand{\lar}{\lambda_R}
\newcommand{\lal}{\lambda_L}
\newcommand{\larl}{\lambda_{R,L}}
\newcommand{\lalr}{\lambda_{L,R}}
\newcommand{\blar}{\ov{\lambda}{}_R}
\newcommand{\blal}{\ov{\lambda}{}_L}
\newcommand{\blarl}{\ov{\lambda}{}_{R,L}}
\newcommand{\blalr}{\ov{\lambda}{}_{L,R}}

\newcommand{\tgamma}{\wt{\gamma}}
\newcommand{\btgamma}{\ov{\tgamma}}
\newcommand{\bpsi}{\ov{\psi}{}}
\newcommand{\bphi}{\ov{\phi}{}}
\newcommand{\bxi}{\ov{\xi}{}}

\newcommand{\ff}{\mc{F}}
\newcommand{\bff}{\ov{\mc{F}}}

\newcommand{\eer}{\epsilon_R}
\newcommand{\eel}{\epsilon_L}
\newcommand{\eerl}{\epsilon_{R,L}}
\newcommand{\eelr}{\epsilon_{L,R}}
\newcommand{\beer}{\ov{\epsilon}{}_R}
\newcommand{\beel}{\ov{\epsilon}{}_L}
\newcommand{\beerl}{\ov{\epsilon}{}_{R,L}}
\newcommand{\beelr}{\ov{\epsilon}{}_{L,R}}

\newcommand{\bi}{{\bar \imath}}
\newcommand{\bj}{{\bar \jmath}}
\newcommand{\bk}{{\bar k}}
\newcommand{\bl}{{\bar l}}
\newcommand{\bm}{{\bar m}}

\newcommand{\hsigma}{{\hat{\sigma}}}
\newcommand{\ww}{\tilde{\mc{W}}{}_\text{eff}}

\begin{document}

{\bf\centering
\large\underline{A note on vacuum values of the superpotential in CP(2)}}
%\vspace{0.5cm}

\section*{CMS Condition in CP(2)}
	Consider CP(2) model for simplicity.
	It is enough to focus on the vacua $ \sigma_1 $ and $ \sigma_0 $. 
	The superpotential is
\beq
\label{Weff}
	\mc{W}_\text{eff} (\hsigma) ~~=~~-\, \frac{1}{2}i \left(\frac{\theta}{2\pi}\right)\, \hsigma ~+~
		\frac{1}{4\pi} \sum_j (\hsigma - m_j)\, 
					\left\{ \ln \frac{\hsigma - m_j}{\Lambda} ~-~ 1 \right\}\,.
\eeq
	Let us denote
\beq
	\ww (\hsigma) ~~=~~ 4\pi\, \mc{W}_\text{eff} (\hsigma)\,.
\eeq
	Then it can be shown that the superpotential values in the vacua $ \sigma_1 $ and 
	$ \sigma_0 $ are related by
\beq
\label{w1}
	\ww (\sigma_1) ~~=~~ e^{2\pi i /3 } \cdot \ww (\sigma_0) ~+~ 2\pi i \cdot m_1\,,
\eeq
	where $ m_1 ~=~ |m|\,e^{i \pi /3 } $.
	Since this equation is derived in the vicinity of the AD point, 
	it is valid only for sufficiently small $ \sigma $, so that various overhangs
	of phases of $ \hsigma - m_j $ over $ 2\pi i $ do not occur due to the size of $ \hsigma $.
	In fact $ |\sigma| ~<~ m $ is enough.
%%	Since for CP(2) we have to choose $ m_0 $ negative, the latter condition is always obeyed
%%\beq
%%	| \sigma |^3 ~~=~~ |\, \sqrt[3]{ \Lambda^3 ~+~ m^3 } \,|^3 
%%		     ~~=~~ |\, \Lambda^3 ~+~ m^3 \,|  ~~<~~  | m |^3\,,
%%\eeq
%%	as long as $ | m | > \Lambda $.

	Equation \eqref{w1} was derived by taking expression \eqref{Weff} literally. 
	As it is seen, the difference of the superpotential values in $ \sigma_1 $ and
	$ \sigma_0 $ does indeed involve a mass, 
\begin{align}
%
\notag
	\ww (\sigma_1) ~-~ \ww (\sigma_0) & ~~=~~ 
	\ww (\sigma_0) \, \Big( e^{2\pi i /3} ~-~ 1 \Big)  ~+~  2\pi i \cdot |m|\,e^{ i\pi/3 }
	~~=~~ \\[2mm]
%
\label{wdiff}
	& 
	~~=~~
	\ww (\sigma_0) \, \Big(\, \frac{m_1}{m_0}  ~-~ 1 \,\Big)  ~+~  2\pi i \cdot |m|\,e^{ i\pi/3 }\,.
\end{align}
	For the masses to sit on the CMS, this difference needs to be compared to any
	mass difference, {\it e.g.}
\beq
	m_{k+1} ~-~ m_k  ~~\propto~~ e^{2 \pi i/ 3} ~-~ 1\,.
\eeq
	The CMS condition is then
\beq
	\ww (\sigma_1) ~-~ \ww(\sigma_0) ~~\propto~~ 2\pi i\, ( m_{k+1} ~-~ m_k )\,,
\eeq
	with a real constant of proportionality.
	However, it can be seen that equation \eqref{wdiff} in the vicinity of $ \sigma ~=~ 0 $, 
	that is,  $ m ~\to~ \Lambda $, has a trivial solution to the CMS condition,
\beq
	\ww (0) ~~=~~ -\, \frac{2\pi}{\sqrt{3}}\, \Lambda 
		~~=~~ -\, 2\pi i \, \Lambda \cdot \frac{1}{e^{i\pi/3} \,-\, e^{-i\pi/3}}\,.
\eeq
	At this value of $ \ww $, the mass term on the RHS of Eq.~\eqref{wdiff} exactly 
	cancels the $ \ww( \sigma_0 ) $ term, 
	and therefore gives a zero difference of the superpotential values,
\beq
	\ww ( \sigma_1 ) ~~=~~ \ww( \sigma_0 )\,, \qquad \lim ~|\sigma| ~\to~ 0\,.
\eeq
	The same value of $ \ww $ is obtained by approaching the AD point in the mirror representation
\begin{align}
\notag
%
	\hat{\mc{W}} & ~~=~~
		-\, \frac{\Lambda}{4\pi} 
			\lgr   X \,+\, Y \,+\, \frac{1}{X Y}  ~-~  \right. \\
%
\label{wmirror}
	&
	\left.
	\qquad\qquad\qquad
			 ~-~  \left\{ \frac{m_1}{\Lambda}\, \ln\, X ~+~
				      \frac{m_2}{\Lambda}\, \ln\, Y ~+~
				      \frac{m_3}{\Lambda}\, 
					\ln\, \frac{1}{X Y} \right\} \rgr.
\end{align}
	Here one of the AD points is $ X ~=~ 1 $, $ Y ~=~ e^{2\pi i /3} $.
	Substitution of this into \eqref{wmirror} gives
\beq
	4\pi \cdot
	\mc{W}_\text{mirror}^\text{CP(2)} ~~=~~ -\, \frac{2\pi}{\sqrt{3}}\, \Lambda\,.
\eeq

\section*{General Discussion}

	The central argument in the derivation of Eq.~\eqref{w1} is that the vacuum values
	seem not to need any further adjustment from what is obtained from Eq.~\eqref{Weff}.
	The $ Z_{2N} $ symmetry manifests itself in the $ Z_N $ structure of the vacua, in each of which
	the former is broken to $ Z_2 $ (I guess this stays true as long as the masses are
	on the circle).

	In the massless CP($N-1$) theory, 
\beq
	-2\, \mc{W}_\text{eff} (\hsigma) ~~=~~ i\, \tau\hsigma ~-~ 
		\frac{N}{2\pi}\, \hsigma \, 
				      \big( \ln (\hsigma/\mu) ~-~ 1 \big)\,,
\eeq
	it appears that in any vacuum, the logarithm term exactly cancels the $ i\, \tau\hsigma $ term, 
	and the superpotential then is proportional to $ \sigma_j $, and, 
	accordingly, its values are the $ Z_N $ roots of unity.
	For example, in CP(1) this is reproduced in the mirror representation via
\begin{align}
%
\label{mirrvv}
	\mc{W}_\text{mirror}^\text{CP(1)}\bigg|_\text{vacuum (1)} & ~~=~~
		-\, \frac{\Lambda}{4\pi} 
			\lgr \sqrt{ (\Delta m / \Lambda)^2 + 4\, } ~~-~~ \frac{\Delta m}{\Lambda}\, \ln\, X^{(1)} \rgr,
	\\
%
\notag
	\mc{W}_\text{mirror}^\text{CP(1)}\bigg|_\text{vacuum (2)} & ~~=~~
		-\, \frac{\Lambda}{4\pi}
			\lgr -\, \sqrt{ (\Delta m / \Lambda)^2 + 4\, } ~~+~~ 
			 \frac{\Delta m}{\Lambda}\, \big(\ln\, X^{(1)} ~-~ i\pi \big) \rgr ,
\end{align}
	by putting $ \Delta m ~=~ 0 $.

	In the twisted-massive theory, this does not happen anymore. 
	Special adjustment is needed to place the vacuum values $ \mc{W}_\text{eff} $ on $ Z_N $.
	I am attaching a picture of the plane of $ \sigma $ as the variable of 
	the multi-leaf function $ \mc{W}_\text{eff}(\hsigma) $ for CP(2).
	It would not matter of course in which direction the logarithm branch cuts extend. 
	It is important however, that an adjustment needed to put the vacuum values of the
	superpotential on the circle would place the $ \sigma_j $ vacua on 
	different sheets of the logarithms.
	It then would be impossible to bring the vacua together, since the origin is not the 
	common branch point. 

	It seems favorable then that the vacuum values, and correspondingly, the kink
	masses are determined from Eq.~\eqref{Weff} as it is. 
	The $ Z_{2N} $ symmetry is realized on the structure of the vacua, and 
	in the invariance of kink masses.
	Equation \eqref{w1} is an example of how the vacuum values are related to each other
	in CP(2),
\beq
\label{w1again}
	\ww (\sigma_1) ~~=~~ e^{2\pi i /3 } \cdot \ww (\sigma_0) ~+~ 2\pi i \cdot m_1\,.
\eeq
	The difference of the vacuum values
\beq
\label{diffagain}
	\ww (\sigma_0) \, \Big( e^{2\pi i /3} ~-~ 1 \Big)  ~+~  2\pi i \cdot m_1\,,
\eeq
	despite an explicit $ m_1 $, does not necessarily contradict the fact that 
	the kink masses depend only on the difference between the vacua numbers 
	$ p \,=\, k \,-\, l $ \cite{Dorey:1999zk}
	(according to which one would expect to see $ m_1 \,-\, m_0 $), 
	as the RHS of Eq.~\eqref{diffagain} contains an unevaluated quantity $ \ww (\sigma_0) $.
	Presumably, if this equation is evaluated, one should be able to see that the kink
	masses only depend on $ m_k \,-\, m_l $ and $ k - l $

\begin{thebibliography}{99}
%\cite{Olmez:2007sg}
\bibitem{Olmez:2007sg}
  S.~Olmez and M.~Shifman,
  %``Curves of Marginal Stability in Two-Dimensional CP(N-1) Models with
  %Z_N-Symmetric Twisted Masses,''
  J.\ Phys.\ A  {\bf 40}, 11151 (2007)
  [arXiv:hep-th/0703149].
  %%CITATION = JPAGB,A40,11151;%%

%\cite{Dorey:1998yh}
\bibitem{Dorey:1998yh}
  N.~Dorey,
  %``The BPS spectra of two-dimensional supersymmetric gauge theories with
  %twisted mass terms,''
  JHEP {\bf 9811}, 005 (1998)
  [arXiv:hep-th/9806056].
  %%CITATION = JHEPA,9811,005;%%

%\cite{Hanany:1997vm}
\bibitem{Hanany:1997vm}
  A.~Hanany and K.~Hori,
  %``Branes and N = 2 theories in two dimensions,''
  Nucl.\ Phys.\  B {\bf 513}, 119 (1998)
  [arXiv:hep-th/9707192].
  %%CITATION = NUPHA,B513,119;%%

%\cite{Dorey:1999zk}
\bibitem{Dorey:1999zk}
  N.~Dorey, T.~J.~Hollowood and D.~Tong,
  %``The BPS spectra of gauge theories in two and four dimensions,''
  JHEP {\bf 9905}, 006 (1999)
  [arXiv:hep-th/9902134].
  %%CITATION = JHEPA,9905,006;%%

\end{thebibliography}



\end{document}

